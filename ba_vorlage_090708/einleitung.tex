\chapter{Einleitung}
Da sich im Universum viele Objekte befinden, die Photonen einer so hohen Energie aussenden, dass diese nicht thermischen Ursprungs sein können, ist die Untersuchung dieser Quellen interessant. Besonders an der hohen Energie dieser Quellen ist zu dem, dass sie die Energien übersteigt, die man in irdischen Teilchenbeschleunigern erreichen kann, womit man durch die Untersuchung dieser Quellen auf Hinweise neuer Physik stoßen kann.\\
Zur Detektion auf der Erde verwendet man Tscherenkow-Teleskope, welche die hochenergetischen Photonen indirekt nachweisen können. Aktuell existieren drei große Projekte (HESS, MAGIC und VERITAS),die durch das Cherenkow Telescope Array (CTA) ersetzt werden sollen. Die Idee des CTAs ist es durch jeweils einen System von Teleskopen auf Nord- und Südhalbkugel den gesamten Nachthimmel abzudecken und durch den Einsatz von drei verschieden großen Teleskoptypen einen großen Energiebereich abzudecken.\\
Ein Prototyp für das mittelgroße Teleskop (MST - Medium Sized Telescope) ist in Berlin-Adlershof errichtet worden, um die unter anderem das Pointing, also die richtige Ausrichtung des Teleskops, zu testen. Dazu befinden sich in der Mitte des Reflektors drei CCD-Kameras zum Testen zweier verschiedener Konzepte. Die Idee ist dabei, dass sowohl der Detektor als auch der Nachthimmel beobachtet wird. Für das erste Konzept wird eine Kamera mit großem Gesichtsfeld verwendet, die den Detektor und den ihn umgebenden Nachthimmel beobachtet. Das zweite Konzept verwendet zwei Kameras, wobei eine nur den Detektor und die zweite nur den Nachthimmel ablichtet. Dazu ist diese schräg zu den anderen Kameras montiert, sodass sich der Detektor nicht mehr im Gesichtsfeld befindet. Durch den Winkel zwischen optischer Achse des Teleskops und dieser Kamera wird ein Modell benötigt, das die Beziehung der Koordinaten der Kamera und des Teleskops beschreibt.