\chapter{Einleitung}
Da sich im Universum viele Objekte befinden, die Photonen einer so hohen Energie aussenden, dass diese nicht thermischen Ursprungs sein können, ist die Untersuchung dieser Quellen interessant. Besonders an der hohen Energie dieser Quellen ist zu dem, dass sie die Energien übersteigt, die man in irdischen Teilchenbeschleunigern erreichen kann, womit man durch die Untersuchung dieser Quellen auf Hinweise zu neuer Physik hoffen kann.\\
Zur Detektion auf der Erde verwendet man Tscherenkow-Teleskope, welche die hochenergetischen Photonen indirekt nachweisen können. Aktuell existieren drei große Projekte (HESS, MAGIC und VERITAS),die durch das Cherenkow Telescope Array (CTA) ersetzt werden sollen. Die Idee des CTAs ist es durch jeweils einen System von Teleskopen auf Nord- und Südhalbkugel den gesamten Nachthimmel abzudecken und durch den Einsatz von drei verschieden großen Teleskoptypen einen großen Energiebereich abzudecken.\\
Ein Prototyp für das mittelgroße Teleskop (MST - Medium Sized Telescope) wurde in Berlin-Adlershof errichtet um die verschiedene Systeme wie das Pointing, das heißt die richtige Ausrichtung des Teleskops, zu testen. Dazu befinden sich in der Mitte des Reflektors drei CCD-Kameras um zwei verschiedene Konzepte zu testen. Die Idee dabei ist, dass sowohl der Detektor als auch der Nachthimmel beobachtet wird. Eine Kamera hat ein großes Gesichtsfeld und lichtet den Detektor und den ihn umgebenen Nachthimmel ab. Eine weitere nur den Detektor und die letzte ist schräg zu den anderen montiert um den Nachthimmel ohne Detektor abzulichten. Durch den Winkel zwischen optischer Achse des Teleskops und dieser Kamera wird ein Modell benötigt, das die Beziehung der Koordinaten der Kamera und des Teleskops beschreibt.