\chapter{gamma-Astronomie}
Die Astronomie ist die Wissenschaft des Universums und beschreibt die Bewegung und Eigenschaften von Himmelskörpern wie Planeten oder Galaxien, interstellarer Materie und Strahlung. Daten ueber diese Objekte werden in der Regel ueber die Detektion von Photonen (sowohl im sichtbaren als auch im unsichtbaren Bereich) gesammelt. Photonen haben den Vorteil, dass sie nicht wie geladene Teilchen durch elektromagnetische Felder abgelenkt werden und deutlich leichter zu detektieren sind als Neutrinos. Die VHE- (very high energy) Astronomie beschäftigt sich mit Strahlungsquellen, die so hochenergetisch sind, dass sie nicht thermischen Ursprungs sind. Diese liegen in der Größenordnung von 100 GeV und darüber.%cite Design concept

\section{Quellen und Entstehung von hochenergetischer Strahlung}
%ein bisschen was zu quellen
Die Entstehung der hochenergetischen Strahlung laesst sich durch zwei Prinzipien erklaeren:
\subsection{bottom up}
Hierbei werden hochenergetische Photonen durch die Wechselwirkung von hochrelativistischen Teilchen erzeugt. Durch fliegen zum Beispiel geladene Teilchen durch Materie wie eine Gaswolke so werden durch Bremsstrahlung bei Annaeherung der Teilchen an die Atome des Mediums Photonen abgestrahlt. Einen Speziallfall der Bremsstrahlung erhaelt man, wenn man die geladen Teilchen durch die Lorentzkraft in Magnetfeldern ablenkt. Die emittierte Strahlung wird in diesem Fall Synchrotronstahlung genannt. Desweiteren koennen Photonen auch durch den inversen Comptoneffekt einen Teil des Impulses aufnehmen. %vielleicht ein paar werte groessenordnungen
\subsection{top down}
In diesem Schema wird die Strahlung durch Zerfälle von massiven Teilchen freigesetzt. Diese Teilchen können auch zur dunklen Materie gehören, sodass man hoffen kann, durch diese Forschung Fortschritte auf diesem Gebiet zu machen.

\section{Detektion von Strahlung}
Da die Erdatmosphaere nicht fuer jede Strahlung durchsichtig ist, bietet sich die Moeglichkeit Teleskope im Weltall zu platzieren oder durch indirekte Messsung am Boden Daten zu sammeln. Im folgenden wird sich mit der zweiten Variante beschaeftigt.

\subsection{Luftschauer}
Treten hochenergetische Photonen in die Materie ein, so wechselwirken sie mit dieser ueber Paarbildung. Das entstehende Elektron bzw Positron verliert daraufhin Energie durch Bremstrahlung, worauf die entstehenden Photonen wieder durch Paarbildung wechselwirken koennen. Somit steigt die Anzahl der Teilchen exponentiell an und die durchschnittliche Energie nimmt exponentiell ab, bis die Teilchen ioniserend sind und der Schauer verschwindet. Die entstehenden Teilchen lassen sich nicht direkt nachweisen, da der Schauer bereits in einer Hoehe von ca 10km verschwindet. %bild zur veranschaulichung

\subsection{Cherenkov Strahlung}
Cherenkov Strahlung tritt auf, wenn geladene sich Teilchen in Materie schneller als Photonen bewegen und lässt sich analog zum Überschallknall erklären. Das geladene Teilchen polarisiert auf seiner Trajektorie die einzelnen Atome, die somit Licht sphaerisch abstrahlen. Da sich das Teilchen allerdings schneller als das Licht bewegt, entsteht ein Kegel konstruktiver Interferenz. Somit entsteht ein Lichtblitz, der sich kegelfoermig mit dem Oeffnungswinkel %bild zur interferenz
\begin{align}
\theta = \arccos\left(\frac{1}{\beta n}
\end{align}
ausbreitet.

\subsection{Detektion mit Teleskopen}
Da aufgrund der Atmosphaere weder das primaere Photon noch die Teilchen des Luftschauers detektiert werden koennen, versucht man die Cherenkovstrahlung, die durch den Luftschauer entsteht zu detektieren. Dazu muss eine grosse Flaeche abgedeckt werden, da selbst bei vertikaler Einstrahlung der Schauer einen Durchmesser von ca 250m haben kann %design concept
