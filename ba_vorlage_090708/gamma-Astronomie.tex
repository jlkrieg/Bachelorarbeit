\chapter{gamma-Astronomie}
Die Astronomie ist die Wissenschaft des Universums und beschreibt die Bewegung und Eigenschaften von Himmelskörpern wie Planeten oder Galaxien, interstellarer Materie und Strahlung. Betrachtete man frueher nur Licht im optisch sichtbaren Bereich, so sind im 20. Jahrhundert einige zusatliche Quellen dazugekommmen. Dazu zaehlen die von Viktor HESS durch Ballonversuche entdeckte kosmische Strahlung, die Roentgen-/bzw die Gammastrahlung sowie die Neutrinoastronomie. Die Gammaastronomie beschaftigt sich mit Photonen im Bereich von bis . Photonen haben den Vorteil, dass sie nicht wie geladene Teilchen durch elektromagnetische Felder abgelenkt werden und somit ihre Quelle leichter detektiert werden kann. Zudem sind sie auch noch deutlich leichter zu detektieren sind als Neutrinos. Da die Energie dieser Photonen so hoch ist, koennen sie nicht thermischen Ursprungs sein sondern kommen aus anderen Quellen, deren Untersuchung das Ziel der Hochenergie-Gamma-Astronomie ist.

%cite Design concept

\section{Entstehung hochenergetischer Strahlung}
\subsection{inverser Comptoneffekt}
Durch den Comptoneffekt koennen hochenergetische Photonen einen Teil ihres Impulses und Energie an ein freies Elektron uebergeben. Dieser Prozess kann auch invers ablaufen und somit kann ein niederenergetisches Photon, zum Beispiel aus dem kosmischen Mikrowellenhintergrund (E), durch einen Stoss mit einem Elektron eine hohe Energie bekommen.
\subsection{Zerfall von schweren Teilchen}
Zerfallen schwerere Teilchen in Photonen, so wird die Ruheenergie dieses Teilchens in kinetische Energie der Photonen umgewandelt. Ein Beispiel hierfuer ist der Zerfall des neutralen Pions, die haufig bei der Kollision von Atomkernen entstehen. Das Pion hat eine Ruhemasse von 135MeV 
%cite pdg
und zerfaellt fast ausschliesslich in zwei Photonen, die dann eine Energie von ungefaehr 68 MeV haben.
\subsection{Materie-Antimaterie-Annihilation}
Bei der Kollision von Materie mit Antimaterie vernichten sich die beiden Teilchen und es entstehen Neue. Dies koennen Photonen sein oder Teichen, die wiederum in Photonen zerfallen. Ein prominetes Beispiel hierfuer ist die Elektron-Positron-Annihilation. Besitzen die beiden Teilchen keine kinetische Energie, so zerfallen sie in zwei Photonen mit der Energie E=511keV.
\subsection{Bremstrahlung}
Durchfliegen hochenergetische Teilchen Materie, so kann es vorkommen, dass diese eng an den Atomen vorbeifliegen und abgelenkt werden. Durch diese Ablenkung werden Photonen abgestrahlt.
\subsection{unbekannte Effekte}
Hochenergetische Photonen koennen auch nach Prinzipien erzeugt werden, die man heute noch nicht versteht. So koennte es moeglich sein, dass hochenergetische Photonen durch den Zerfall von Partikeln der dunklen Materie stammen. Die Supersymmetrie sagt zum Beispiel den Zerfall von schweren WIMPS in Photonen vorraus. Durch Detektion solcher Ereignisse liesse sich auf neue Physik schliessen.

\section{Quellen hochenergetischer Strahlung}
Ziel VHE-Astronomie ist es die Quellen hochenergetischer Gammastrahlung zu erforschen. Folgende Quellen sind bekannt:
\subsection{Supernova Ueberreste}
\subsection{Pulsare}
\subsection{Quasare}
\subsection{Stelare}
\subsection{Aktive Galaxien}
\subsection{Binaere Systeme}
\subsection{Gamma Ray Bursts}


\section{Detektion von Strahlung}
Prinzipiell laesst sich zwischen bodengestuetzter und satellitengestuetzter Gammaastronomie unterscheiden. Durch den Einsatz von Satelliten vermeidet man den stoerenden Einfluss der Erdatmosphaere, muss dafuer Abstriche in der Groesse der Detektoren machen und mit hohen Kosten kalkulieren. Hier soll sich nur mit der bodengestuetzten Variante beschaeftigt werden.

\subsection{Luftschauer}
Treten hochenergetische Photonen in die Materie ein, so wechselwirken sie mit dieser ueber Paarbildung. Das entstehende Elektron bzw Positron verliert daraufhin Energie durch Bremstrahlung, worauf die entstehenden Photonen wieder durch Paarbildung wechselwirken koennen. Somit steigt die Anzahl der Teilchen exponentiell an und die durchschnittliche Energie nimmt exponentiell ab, bis die Teilchen ioniserend sind und der Schauer verschwindet. Die entstehenden Teilchen lassen sich nicht direkt nachweisen, da der Schauer bereits in einer Hoehe von ca 10km verschwindet. %bild zur veranschaulichung

\subsection{Cherenkov Strahlung}
Cherenkov Strahlung tritt auf, wenn geladene sich Teilchen in Materie schneller als Photonen bewegen und lässt sich analog zum Überschallknall erklären. Das geladene Teilchen polarisiert auf seiner Trajektorie die einzelnen Atome, die somit Licht sphaerisch abstrahlen. Da sich das Teilchen allerdings schneller als das Licht bewegt, entsteht ein Kegel konstruktiver Interferenz. Somit entsteht ein Lichtblitz, der sich kegelfoermig mit dem Oeffnungswinkel %bild zur interferenz
\begin{align}
\theta = \arccos\left(\frac{1}{\beta n}
\end{align}
ausbreitet.

\subsection{bodengestuezte Detektion mit Teleskopen}
Da aufgrund der Atmosphaere weder das primaere Photon noch die Teilchen des Luftschauers detektiert werden koennen, versucht man die Cherenkovstrahlung, die durch den Luftschauer entsteht zu detektieren. Dazu muss eine grosse Flaeche abgedeckt werden, da selbst bei vertikaler Einstrahlung der Schauer einen Durchmesser von ca 250m haben kann %design concept
