\chapter{$\gamma$-Astronomie} 
Die Astronomie ist die Wissenschaft des Universums und beschreibt die Bewegung und Eigenschaften von Himmelskörpern wie Planeten oder Galaxien, interstellarer Materie und Strahlung. Betrachtete man frueher nur Licht im optisch sichtbaren Bereich, so sind im 20. Jahrhundert einige zusatliche Quellen dazugekommmen. Dazu zaehlen die von Viktor HESS durch Ballonversuche entdeckte kosmische Strahlung, die Roentgen-/bzw die Gammastrahlung sowie die Neutrinoastronomie. Die Gammaastronomie beschaftigt sich mit Photonen im Bereich von bis . Photonen haben den Vorteil, dass sie nicht wie geladene Teilchen durch elektromagnetische Felder abgelenkt werden und somit ihre Quelle leichter detektiert werden kann. Zudem sind sie auch noch deutlich leichter zu detektieren sind als Neutrinos. Da die Energie dieser Gammastrahlung so hoch ist, koennen sie nicht thermischen Ursprungs sein sondern kommen aus anderen Quellen, deren Untersuchung das Ziel der Hochenergie-Gamma-Astronomie (VHE - Very High Energy) ist.

%cite Design concept

\section{Entstehung hochenergetischer Strahlung}

\begin{description}
\item[inverser Comptoneffekt]\hfill \\
Durch den Comptoneffekt koennen hochenergetische Photonen einen Teil ihres Impulses und Energie an ein freies Elektron uebergeben. Dieser Prozess kann auch invers ablaufen und somit kann ein niederenergetisches Photon, zum Beispiel aus dem kosmischen Mikrowellenhintergrund (E), durch einen Stoss mit einem Elektron eine hohe Energie bekommen.
\item[Bremstrahlung]\hfill \\
Durchfliegen hochenergetische Teilchen Materie, so kann es vorkommen, dass diese eng an den Atomen vorbeifliegen und abgelenkt werden. Durch diese Ablenkung werden Photonen abgestrahlt.
\item[Zerfälle und Annihilation]\hfill \\ 
Hochenergetische Photonen können auch durch Zerfälle massiver Teilchen entstehen, wobei die Ruhemasse des Teilchen in kinetischer Energie der Photonen umgewandelt wird. So zerfällt das neutrale Pion zum Beispiel zu 98,8\% \cite{PDG} in zwei Photonen und setzt dabei eine Ruhemasse von $E_0=135MeV$ \cite{PDG} um. Eine weitere Möglichkeit ist die Annihilation von Materie und Antimaterie. Auch hier wird die Ruheenergie der Teilchen in kinetische Energie umgewandelt. So entstehen bei der Elektron-Positron-Annihilation zwei Photonen mit der Energie $E=511keV$.
Diese Energien liegen allerdings noch weit unter der Grenze der VHE.

\end{description}

\section{Quellen hochenergetischer Strahlung}
Ziel VHE-Astronomie ist es die Quellen hochenergetischer Gammastrahlung zu erforschen. Folgende Quellen sind bekannt:

\begin{description}
\item[Supernova Uebereste]\hfill \\
Hat ein Stern mit ausreichender Masse seinen Wasserstoff- und Heliumvorat verbrannt und somit sein Lebensende erreicht, kollabiert dieser und es entsteht ein kompaktes Objekt, wie zum Beispiel ein Neutronenstern
\item[Schwarze Loecher]\hfill \\
Schwarze Löcher sind Objekte mit einer Gravitationskraft, die so stark ist, dass auch Photonen, die sich hinter dem Ereignishorizont befinden nicht entkommen können. Durch die starke Anziehung entsteht eine Akkretionsscheibe in der große elektromagnetische Felder herrschen, durch die wiederum hochenergetische Photonen entstehen können.
\item[Aktive Galaxien und aktive galaktische Kerne]\hfill \\
Da schwarze Löcher zu den hellsten Gammaquellen gehören und sich in der Regel im Zentrum einer Galaxie befinden, können diese überstrahlen. Häufig werden diese als stellare Objekte wargenommen. AGNs haben eine Masse von rund 100 Millionen Sonnenmassen. Zu den AGNs gehören auch Blazare und Quasare.
\item[Pulsare]\hfill \\
Kollabieren die Überreste einer Supernova von einem Durchmesser von ca $10^6km$ auf ca $20km$ ensteht ein Neutronenstern, der sich aufgrund der Drehimpulserhaltung sehr schnell dreht. Solche Konstrukte nennt man Pulsare. Durch die schnelle Rotation entstehen starke elektromagnetische Felder, die geladene Partikel beschleunigen können. Pulsare strahlen ungefähr $10^42eV/s$ ab.
\item[Binäre Systeme]\hfill \\ 
Befindet sich ein Neutronenstern oder Pulsar in einem System mit einem normalen Stern, entsteht durch absaugen von Materie eine Akkretitionsscheibe um den Neutronenstern beziehungsweise um den Pulsar. Da die Materie in diesem System durch Gravitation beschleunigt wird, werden Energien der Größenordnung $10^19 eV$ erzeugt. 
\item[Dunkle Materie]\hfill \\
Dunkle Materie gehört zu den größten Fragen der heutigen Physik. Man hofft, dass durch VHE-Astronomie Erkenntnisse gewonnen werden, die Theroien zur dunklen Materie (wie zum Beispiel WIMPS) entweder bestätigen oder widerlegen können.
\end{description}


\section{Detektion von Strahlung}
Prinzipiell laesst sich zwischen bodengestuetzter und satellitengestuetzter Gammaastronomie unterscheiden. Durch den Einsatz von Satelliten vermeidet man den stoerenden Einfluss der Erdatmosphaere, muss dafuer Abstriche in der Groesse der Detektoren machen und mit hohen Kosten kalkulieren. Hier soll sich nur mit der bodengestuetzten Variante beschaeftigt werden, die günstiger ist und nicht in der Größe beschränkt ist. Dazu verwendet man sogenannte IACTS (Imaging Atmosphaeric Cherenkov Telescopes), die die Strahlung nur indirekt detekieren.

\subsection{Luftschauer}
Die Atmosphäre ist nur für Photonen im optischen und radio Bereich durchsichtig. Treten hochenergetische Photonen in die Atmosphäre ein, wechselwirken sie mit dieser über Paarbildung. Das entstehende Elektron bzw Positron ist ebenfalls hochenergetisch und verliert hauptsächlich durch Bremstrahlung Energie, worauf die entstehenden Photonen wieder durch Paarbildung wechselwirken koennen. Die Strahlungslängen für Paarbildung und Bremsstrahlung sind ungefähr gleich lang, sodass die Anzahl der Teilchen mit absteigender Höhe exponentiell zunimmt, wohingegend die durchschnittliche Energie der Teilchen exponentiell abnimmt. Der Luftschauer endet, in einer Höhe von ungefähr 10km \cite{iwas}, wenn die Teichen niederenergetisch sind und die restliche Energie über Ionisation verlieren.
\begin{equation}
E_n=\frac{E_0}{2^n}
\end{equation}\\
Neben den oben beschriebenen elektromagnetischen Schauern existieren noch hadronische und myonische Schauer. Hadronische Schauer entstehen wenn hochenergetische Hadronen in die Atmosphaere eindringen. Durch die Wechselwirkung von Hadronen entstehen haufig Pionen, die wiederum in Photonen zerfallen, wodurch wiederum ein elekromagnetischer Schauer entsteht, der allderdings einen anderen Ursprung hat. Entstehen Myonen in einem Schauer, so besteht das Problem, dass diese kaum Energie abgeben und bei hoher Geschwindigkeit den Erdboden erreichen. Somit gibt nur ein Teil des Schauers die Energie ab und die Messung weicht von der Realitaet ab.

\subsection{Cherenkov Strahlung}
Cherenkov Strahlung tritt auf, wenn geladene sich Teilchen in Materie schneller als die Lichtgeschwindigkeit in diesem Medium bewegen. Hierbei polarisiert das geladene Teilchen auf seiner Trajektorie die einzelnen Atome, die Licht sphaerisch abstrahlen. Wäre das Teilchen langsamer als die Ausbreitungsgeschwindigkeit in diesem Medium, würden die Wellen destruktiv interferrieren und man würde keine makroskopischen Effkte beobachten. Da sich das Teilchen allerdings schneller als das Licht bewegt, entsteht ein Kegel konstruktiver Interferenz, und ein Lichtblitz breitet sich kegelfoermig mit dem Oeffnungswinkel %bild zur interferenz
\begin{equation}
\theta = \arccos\left(\frac{1}{\beta n}\right) \label{eq:cherenkow}
\end{equation}\\
aus. Fuer Luft (in Bodennaehe) ergibt sich somit ein maximaler Oeffnungswinkel von. Da allerdings die Dichte der Luft in der relevanten Höhe kleiner ist, ist auch der Brechungsindex näher an 1 und der Cherenkovwinkel beträgt noch ungefähr $\theta = 1^{\circ}$\cite{Grupen}
\begin{figure}[htbp]
\centering
\includegraphics[width=0.7\textwidth]{Images/cherenkow.png}
\caption{Der Cherenkoweffekt: Ein geladenes Teilchen durchfliegt ein dielektrisches Medium mit einer Geschwindigkeit ueber der der Lichtgeschwindigkeit im Medium und erzeugt Wellenfronten.}
\label{img:cherenkow}
\end{figure}
Aus dem Winkel lässt sich die Geschwindigkeit des Teilchens rekonstruieren und bei bekannter Masse des Teilchens (in elektromagnetischen Schauern entstehen Elektronen als geladene Teilchen) auch der Impuls und die Energie.

\subsection{Bodengestützte Detektion der Cherenkovstrahlung}
Ziel der bodengestützten Variante ist es das Cherenkovlicht der sekundären Teilchen aus dem elektromagnetischen Schauer zu detektieren. Bei einem Cherenkovwinkel von $\theta=1^\{circ}$ in 10km Höhe und senkrechter Einstrahlung ergibt sich ein Lichtpool am Boden mit einem Durchmesser von $250m$. Somit müssen effektive Flächen in der Größenordnung von $10^4$ bis $10^5m^2$ abgedeckt werden um den gesamten Schauer zu detektieren. Typischerweise entstehen bei einem VHE-Photon $10^8-10^9$ Photonen, die innerhalb von wenigen Nanosekunden abgegeben werden. Am Boden hat man somit typische Intenstäten von $10^3\frac{1}{m^2}$ die detektiert werden müssen.
Dazu verwendet man Abbildende Cherenkovteleskope (IACTS - Imaging Atmosphaeric Cherenkov Telescops), die aus einem Reflektor und einem Detektor bestehen. Der Reflektor besteht aus einem oder häufig aus mehreren Spiegeln, die das Cherenkovlicht in der Brennebene bündeln. Bei großen Teleskopen muss der Reflektor parabolisch sein und bei kleineren wird darauf häufig aus Kostengründen verzichtet, da jeder einzelne Spiegel eine individuelle Brennweite braucht.
Als Detektor wird eine Cherenkovkamera verwendet, die eine typische Auflösung von 2000 Pixeln und Zeitaulösung $10ns$\cite{CherenkovCam} hat. Das die Auflösung im Vergleich zu CCD Kameras eher gering ist, liegt daran, dass der Detektor sehr wenige Photonen in einer sehr kurzen Zeit detektiert werden müssen. Dazu verwendet man Photomultiplier.
Aus den aufgenommenen Daten laesst sich mit Hilfe von Monte-Carlo-Simulationen die Richtung und die Energie des detektierten Photons rekonstruieren. 

\begin{figure}[htbp]
\centering
\includegraphics[width=\textwidth]{Images/detection.png}
\caption{Detektion hochenergetischer Strahlung: Das in die Atmosphaere eintretende Photon erzeugt einen elektromagnetischen Luftschauer (oben rechts) der widerum Cherenkovlicht erzeugt, welches am Boden mit IACTs detektiert werden kann. Ein Detektionsbild ist unten links zu sehen.}
\label{img:detection}
\end{figure}
