\chapter{gamma-Astronomie}
Die Astronomie ist die Wissenschaft des Universums und beschreibt die Bewegung und Eigenschaften von Himmelskörpern wie Planeten oder Galaxien, interstellarer Materie und Strahlung. Die VHE- (very high energy) Astronomie beschäftigt sich mit Strahlungsquellen, die so hochenergetisch sind, dass sie nicht thermischen Ursprungs sind. Diese liegen in der Größenordnung von 100 GeV und darüber.%cite Design concept
Die Ursache dieser Strahlung kann prinzipiell durch zwei verschiedene Konzepte beschrieben werden:
\section{bottom up}
Hierbei werden hochenergetische Photonen durch die Wechselwirkung von hochrelativistischen Teilchen mit Materie (zum Beispiel Gaswolken), Magnetfeldern und anderer elektromagnetischer Strahlung erzeugt. In Magnetfeldern und in Materie können diese Teilchen, sofern sie geladen sind, abgelenkt werden (zum Beispiel durch Bremstrahlung) und verlieren dadurch Energie (durch sogenannte Synchrotronstrahlung). Bei der Wechselwirkung der relativistischen Teilchen mit Photonen kann es zu einem iversen Comptoneffekt kommen, wodurch diese Photonen Teile des Impulses aufnehmen können.
\section{top down}
In diesem Schema wird die Strahlung durch Zerfälle von massiven Teilchen freigesetzt. Diese Teilchen können auch zur dunklen Materie gehören, sodass man hoffen kann, durch diese Forschung Fortschritte auf diesem Gebiet zu machen.

\section{Cherenkov Strahlung}
Cherenkov Strahlung tritt auf, wenn geladene sich Teilchen in Materie schneller als Photonen bewegen und lässt sich analog zum Überschallknall erklären. Das geladene Teilchen polarisiert auf seiner Trajektorie die einzelnen Atome, die somit Licht abstrahlen. Da sich das Teilchen allerdings schneller als das Licht bewegt, kann keine destruktive Interferenz auftreten, sodass sich das Licht in einem Kegel ausbreitet. Der Kosinus des Öffnungswinkels ist invers proportional zur Geschwindigkeit des Teilchens und somit auch zum Impuls