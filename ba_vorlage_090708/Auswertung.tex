\chapter{Anwendung des Pointingmodells auf Daten des Teleskops}
\label{ch:auswertung}
Um das in Kapitel \ref{ch:pointing} entwickelte Pointingmodell zu testen, wurde mit dem Prototyp des MST ein Datensatz aufgenommen um
\section{Datensatz}
Zur Analyse wurde der in der Nacht vom 4. auf den 5. Juli 2018 im Zeitraum von 21:00 UTC bis 1:45 UTC mit dem Prototypen des MST in Berlin-Adlershof aufgenommene Datensatz "run341" verwendet. Dazu wurden 105 Postionen beobachtet wobei das Teleskop anfangs Richtung Zenit stand und sich dann in einer abwärtslaufenden Spirale befand.
\begin{figure}[htbp]
\centering
\includegraphics[width=\textwidth]{../341/data4.png}
\label{img:record}
\caption{Die am Teleskop eingestellten Position zur Datenaufnahme}
\end{figure}
An jeder Position wurden 4 Bilder aufgezeichnet, wovon zwei eine Belichtungszeit von 20 Sekunden hatten.
Um die Bilder aufzunehmen, wurde eine Positiondes Nachthimmels eingestellt, die während der Belichtung des Bildes verfolgt wurde. Als Zeitstempel für die jeweiligen Bilder wird die Mitte der Belichtungszeit gewählt. Die jeweils zweiten Bilder werden hier zur Analyse verwendet. Dazu wird zunächst die Software \texttt{astrometry.net} verwendet, die aus den Bildern die jeweiligen Positionen der Sterne liest und diese mit dem bekannten Sternenhimmel vergleicht. Hierdurch lassen sich die Himmelskoordinaten des Mittelpunktes des Bildes sowie die Größe des Bildausschnittes bestimmen.
Die Himmelskoordinaten lassen sich mithilfe des Zeitstempels in das Koordinatensystem des Teleskops umrechnen und aus der Größe des Bildausschnitts lässt sich noch der Pixelscale bestimmen, der angibt, wie groß das Gesichtsfeld eines einzelnen Pixels ist. Da der Pixelscale kameraspezifisch und bekannt ist, lässt sich mit diesem überprüfen, ob das Bild richtig aufgelöst wurde. Von den 105 aufgenommenen Bildern konnten 100 gelöst werden, wobei der Pixelscale wie in Abbildung \ref{img:ps}
\begin{figure}[htbp]
\centering
\includegraphics[width=\textwidth]{../341/histo.png}
\label{img:ps}
\caption{Die Verteilung der von \texttt{astrometry.net} bestimmten Pixelscales}
\end{figure}
zu sehen ist in keinem der Bilder um mehr als $0,04\unit{arcsec}$ abweicht. Somit können alle gelösten Aufnahmen zur Analyse verwendet werden. Zu sehen ist dieser Datensatz in Abbildung \ref{img:dataset}.
\begin{figure}[htbp]
\centering
\includegraphics[width=\textwidth]{../341/data2.png}
\label{img:dataset}
\caption{Zu sehen sind die Abweichungen zwischen gemessenen und eingestellten Koordinaten in Abhängigkeit der gemessenen Koordinaten}
\end{figure}
\section{Bestimmung der optimalen Parameter für die Pointingmodelle}
Um das Pointingmodell auf Konsistenz zu überprüfen, wurde ein Programm mit der C++ Erweiterung ROOT geschrieben, das die optimalen Parameter für die in Kapitel \ref{ch:pointing} Pointingmodelle berechnet. Aus dem Vergleich der mit diesen Modellen vorhergesagten Werte mit den gemessenen Daten lassen sich dann Rückschlüsse über die Qualität der Pointingmodelle schließen. Um die optimalen Parameter zu finden wird die Methode der kleinsten Quadrate angewandt. 
Dazu müssen die Abstände zwischen den vom Pointingmodell mit Parametern $\vec{q}$ bestimmten (Index P) und den gemessenen Positionen (Index M) bestimmt werden. Da alle Positionen, wie in Abbildung \ref{img:coordinates} zu sehen ist, auf einer Kugeloberfläche liegen, lässt sich der Abstand zwischen zwei Punkten durch einen Winkel $\psi$ ausdrücken:
\begin{equation}
\psi=\arccos\left(\sin(el_M)\sin(el_P)+\cos(el_P)\cos(el_M)\cos(az_P-az_M)\right).
\end{equation}
Dazu werden  aus den Tupeln der Messwerte ($M_i$) und den Vorhersagen des Pointingmodells ($P_i$) mit den Parametern $\vec{q}$ Differenzen gebildet. Diese werden quadriert, damit Abweichungen nach unten den gleichen Effekt wie Abweichungen nach oben haben. Die Differenzen aller Werte des Datensatzes werden zu einer Funktion
\begin{equation}
Q(\vec{q})=\sum^N_{i=1}\left(P_i-M_i(\vec{q})\right)^2
\end{equation}
addiert, dei ein Maß für die Abweichung des Modells zu den Messwerten liefert. Diese Funktion kann mit dem eingebauten TMinuit-Paket minimiert werden um die besten Parameter des Modells zu erhalten. Um sich die zu überlegen, ob das Modell mit dem Datensatz vereinbar ist, wird die Größe $\frac{\chi^2}{doF}$ verwendet, die sich aus 
\begin{equation}
\chi^2=\sum^N_{i=1}\frac{\left(P_i-M_i\right)^2}{\sigma_i^2}
\end{equation}
und der Anzahl der Freiheitsgrade
\begin{equation}
doF=N-\textrm{Anzahl der Parameter}
\end{equation}
berechnet. Mit dem Wissen, dass der perfekte Wert für für $\frac{\chi^2}{doF}=1$ ist und der Annahme, dass alle Fehler gleich groß sind, lassen sich die Fehler wie folgt berechnen
\begin{equation}
\sigma=\sqrt{\frac{\sum^N_{i=1}\left(P_i-M_i\right)^2}{doF}}
\end{equation}
Zudem werden noch die Eigenschaften der Parameter bestimmt. Dazu wird ebenfalls mithilfe des TMinuit-Pakets die Kovarianzmatrix ausgegeben, die die Werte
\begin{equation}
COV(X,Y)=\left<\left(X-\left<X\right>\right)\cdot\left(Y-\left<Y\right>\right)\right>
\end{equation}
enthält. Auf der Hauptdiagonale stehen die Varianzen, aus denen sich die Fehler der Parameter bestimmen lassen:
\begin{equation}
\sigma_X=\sqrt{COV(X,X)}=\sqrt{VAR(X)}
\end{equation}
Interessant ist es auch, sich die Korrelation der Parameter untereinander anzusehen. Dazu wird der Korrelationskoeffizient nach Pearson $\rho$ verwendet, der ausschließlich die lineare Korrelation berücksichtigt. Für $\rho=0$ liegt keine lineare Korrelation vor und für $\rho=\pm1$ eine positive beziehungsweise eine negative lineare Korrelation. Der Korrelationskoeffizient lässt sich ebenfalls aus der Kovarianzmatrix berechnen:
\begin{equation}
\rho_{XY}=\frac{COV(X,Y)}{\sigma_X\sigma_Y}
\end{equation}
Zuletzt werden noch die Ergebnisse noch in jeweils vier Plots, die die Differenzen des Modells zu den Messwerten zeigen, graphisch dargestellt.


%Um das Pointingmodell auf Konsistenz zu überprüfen wurde ein Programm in ROOT geschrieben, welches die Differenzen der vom Pointingmodell (Index P) bestimmten Werte mit den gemessenen (Index M) Werten berechnet und analysiert. Da die oben entwickelten Pointingmodelle noch freie Parameter haben, die vom Teleskop abhängen werden diese durch eine Regression der Daten bestimmt. Da das Pointingmodell aus zwei Funktionen besteht (jeweils eine für die Elevation und den Azimut), die jedoch von den gleichen Parametern abhängen, muss hier ein kombinierter Fit durchgeführt werden. Dazu wird eine Hilfsvariable eingeführt, die die Summe der Quadrate der Differenzen von Messwerten und vorhergesagten Werten für feste Werte der freien Parameter bestimmt.
%\begin{equation}
%Q=\sum^N_{i=1}\left(P_i-M_i\right)^2
%\end{equation}
%Durch die Bildung der Quadrate können sich Abweichungen nach oben und unten nicht gegenseitig kompensieren und die Hilfsvariable ist somit ein Maß für die Abweichung von Modell mit den gewählten Parametern und Messwerten. Die besten Parameter erhält man, indem man die Variable minimiert. Dazu wurde die in ROOT integrierte Funktion TMinuit verwendet, die zudem noch die Standardabweichung der Parameter ausgibt. Um die Güte des Modells bestimmen zu können wird noch die Größe $\frac{\chi^2}{doF}$ verwendet um die Fehler der Messwerte zu bestimmen, in denen das Modell mit dem Datensatz übereinstimmt. Die Größe $\frac{\chi^2}{doF}$ ist definiert als
%\begin{equation}
%\frac{\chi^2}{doF}=\sum^N_{i=1}\frac{\left(P_i-M_i\right)^2}{\sigma_i^2}
%\end{equation}
%wobei hier $\sigma_i$ der Fehler des Wertes i ist und $doF$ der Anzahl der Freiheitsgraden entspricht. Diese berechnet sich durch
%\begin{equation}
%doF=N-\textrm{Anzahl der Parameter}
%\end{equation}
%Der perfekte Wert für $\frac{\chi^2}{doF}$ ist 1. Unter der Anahme, dass die Fehler $\sigma$ alle gleich groß sind, lassen sich diese wie folgt berechnen
%\begin{equation}
%\sigma=\sqrt{\frac{\sum^N_{i=1}\left(P_i-M_i\right)^2}{doF}}=\sqrt{\frac{Q}{doF}}
%\end{equation}
%Interessant ist es auch, sich die Korrelation der Parameter untereinander anzugucken. Die Korrelation beschreibt die Beziehung der zwischen einzelnen Parametern. Hier wird dazu der Korrelationskoeffizient nach Pearson verwendet, der ausschließlich die lineare Korrelation berücksichtigt. Um diesen zu berechnen wird zunächst die Korelationsmatrix
%\begin{equation}
%COV(X,Y)=\left<\left(X-\left<X\right>\right)\cdot\left(Y-\left<Y\right>\right)\right>
%\end{equation}
%berechnet. Auf der Hauptdiagonale stehen die Varianzen, aus denen sich die Fehler der Parameter bestimmen lassen:
%\begin{equation}
%\sigma_X=\sqrt{COV(X,X)}=\sqrt{VAR(X)}
%\end{equation}
%Der Korrelationskoeffizient lässt sich nun durch
%\begin{equation}
%\rho_{XY}=\frac{COV(X,Y)}{\sigma_X\sigma_Y}
%\end{equation}\\
%Um die Qualität der jeweiligen Pointingmodelle beurteilen zu können, werden jeweils vier Plots ausgegeben, die die jeweiligen Differenzen der bestimmten und gemessenen Koordinaten angeben
%\begin{equation}
%f=el_M-el_P \quad \textrm{bzw} \quad f=az_M-az_P
%\end{equation}
\section{Anwendung auf das Pointingmodell mit zwei Parametern}
Zunächst soll das in Kapitel \ref{ch:pointing} entwickelte Modell mit zwei Parametern, das einen konstanten Winkel zwischen der optischen Achse des Teleskops und der CCD-Kamera annimmt, unersucht werden.
\subsection{Abhängigkeit der Drivekoordinaten in Abhängigkeit der CCD-Koordinaten}
Begonnen wird mit der Vorhersage der Drive-Koordinaten mithilfe der CCD-Koordinaten. Unter der Annahme des festen Winkels zwischen Drive und Kamera wurden in Kapitel \ref{ch:pointing} die Abhängigkeiten aus den Formeln \ref{eq:elC2D} und \ref{eq:azC2D} vorhergesagt. Verwendet man das oben beschriebene Programm um die optimalen Parameter zu berechnen, so erhält man die Werte aus Tabelle \ref{tab:C2D}.
\begin{table}[htbp]
\centering
\begin{tabular}{rcl}
\toprule
$el_0$ &=& $(-1,20\pm0,32)^{\circ}$\\
$az_0$ &=& $(12,04\pm0,28)^{\circ}$\\
$\sigma$ &=& $0,23^{\circ}$\\
$\rho_{el_0,az_0}$ &=& $0,2618$\\
\bottomrule
\end{tabular}
\label{tab:C2D}
\caption{Die für das Pointingmodell mit zwei Parametern bestimmten Werte, wobei die Drivekoordinaten in Abhängigkeit der CCD-Koordinaten vorhergesagt wurden.}
\end{table}
Die Differenzen zwischen den vom Modell vorhergesagten und den gemessenen Werten sind in Abbildung \ref{img:C2D} dargestellt.
\begin{figure}[htbp]
\centering
\includegraphics[width=\textwidth]{../341/run341C2D.png}
\caption{Gezeigt sind die Differenzen zwischen den vom Zwei-Parameter-Modell mit $az_0=12,04^{\circ}$ und $el_0=-1,20^{\circ}$ vorhergesagten Drive-Koordinaten und den gemessenen Drive-Koordinaten in Abhängigkeit der CCD-Koordinaten}
\label{img:C2D}
\end{figure}

%\begin{table}[htbp]
%\centering
%\begin{tabular}{rcl}
%\toprule
%$el_0$ &=& $(-1,1941\pm0,0098)^{\circ}$\\
%$az_0$ &=& $(12,0507\pm0,0019)^{\circ}$\\
%$\sigma$ &=& $0,23^{\circ}$\\
%$\rho_{el_0,az_0}$ &=& $0,1576$\\
%\bottomrule
%\end{tabular}
%\label{tab:C2D}
%\caption{Die für das Pointingmodell mit 2 Parametern bestimmten Werte}
%\end{table}

\subsection{Abhängigkeit der CCD-Koordinaten in Abhängigkeit der Drivekoordinaten}
Möchte man jedoch die Koordinaten des Drives in Abbhängigkeit der CCD bestimmem, so benötigt man das inverse Modell, welches durch die Formeln \ref{eq:elD2C} und \ref{eq:azD2C} beschrieben wird. Für dieses Modell werden mithlife des Programms die Werte in Tabelle \ref{tab:D2C} bestimmt. Zudem sind die Differenzen zwischen Vorhersage und Modell in Abbildung  \ref{img:D2C} zu sehen.
\begin{table}[htbp]
\centering
\begin{tabular}{rcl}
\toprule
$el0$ &=& $(-1,20\pm0,30)^{\circ}$\\
$az0$ &=& $(12,05\pm0,27)^{\circ}$\\
$\sigma$ &=& $0,19^{\circ}$\\
$\rho_{el_0,az0}$ &=& $-0,00401$\\
\bottomrule
\end{tabular}
\label{tab:D2C}
\caption{Die für das Pointingmodell mit zwei Parametern bestimmten Werte, wobei die CCD-Koordinaten in Abhängigkeit der Drive-Koordinaten vorhergesagt wurden.}
\end{table}


\begin{figure}[htbp]
\centering

\includegraphics[width=\textwidth]{../341/run341D2C.png}
\caption{Gezeigt sind die Differenzen zwischen den vom Zwei-Parameter-Modell mit $az_0=12,05^{\circ}$ und $el_0=-1,20^{\circ}$ vorhergesagten CCD-Koordinaten und den gemessenen CCD-Koordinaten in Abhängigkeit der Drive-Koordinaten}
\label{img:D2C}
\end{figure}

%\end{figure}
%\begin{table}[htbp]
%\centering
%\begin{tabular}{rcl}
%\toprule
%$el0$ &=& $(-1,184\pm0,008)^{\circ}$\\
%$az0$ &=& $(12,048\pm0,004)^{\circ}$\\
%$\sigma$ &=& $0,19^{\circ}$\\
%$\rho_{el_0,az0}$ &=& $-0,4698$\\
%\bottomrule
%\end{tabular}
%\label{tab:D2C}
%\caption{Die Drive-Koordinaten in Abhängigkeit der CCD-Koordinaten im Zwei-Parameter-Modell}
%\end{table}
\section{Anwendung auf das Pointingmodell mit 4 Parametern}
Im Folgenden wird das oben verwendete Modell erweitert, indem man wie in Abschnitt \ref{se:4par} animmmt, dass die eingestellten Koordinaten zu den tatsächlichen um jeweils einen konstanten Wert verschoben sind. Zu den obigen Parametern $el_0$ und $az_0$ kommen somit noch die Parameter $az_1$ für eine konstante Verschiebung im Azimut und $el_1$ für eine konstante Verschiebung in der Elevation hinzu.
\subsection{Abhängigkeit der Drivekoordinaten in Abhängigkeit der CCD-Koordinaten}
Hier wird wie oben mit der Vorhersage der Drive-Koordinaten in Abhängigkeit der CCD begonnnen. Dazu verwendet man die Formeln \ref{eq:elC2D4} und \ref{eq:azC2D4}, für die sich mithilfe des Fit-Programms die Werte aus Tabelle \ref{tab:C2D4} berechnen lassen und die Differenzen zwischen Modell und gemessenen Werten in Abbildung \ref{img:C2D4} darstellen.
\begin{figure}[htbp]
\centering
\includegraphics[width=\textwidth]{../341/run341C2D_4par.png}
\caption{Gezeigt sind die Differenzen zwischen den vom Vier-Parameter-Modell mit $az_0=11,99^{\circ}$, $el_0=-1,22^{\circ}$, $az_1=0,031^{\circ}$ und $el_1=0,0025^{\circ}$ vorhergesagten Drive-Koordinaten und den gemessenen Drive-Koordinaten in Abhängigkeit der CCD-Koordinaten}
\label{img:C2D4}
\end{figure}
\begin{table}[htbp]
\centering
\begin{tabular}{rcl}
\toprule
$el_0$ &=& $(-1,21891\pm 0,03924)^{\circ}$\\
$az_0$ &=& $(11,9854\pm0,2072)^{\circ}$\\
$el_1$ &=& $(0,00251434\pm 0,0001839)^{\circ}$\\
$az_1$ &=& $(0,03080816\pm0,2149)^{\circ}$\\
$\sigma$ &=& $0,23^{\circ}$\\
$\rho_{el_0,az_0}$ &=& $0,8691$\\
$\rho_{el_0,el_1}$ &=& $-0,8417$\\
$\rho_{az_0,az_1}$ &=& $-0,9358$\\
$\rho_{el_0,az_1}$ &=& $-0,8137$\\
$\rho_{az_0,el_1}$ &=& $-0,9709$\\
$\rho_{el_1,az_1}$ &=& $-0,8417$\\
\bottomrule
\end{tabular}
\label{tab:C2D4}
\caption{Die für das Pointingmodell mit vier Parametern bestimmten Werte, wobei die Drive-Koordinaten in Abhängigkeit der CCD-Koordinaten vorhergesagt wurden.}
\end{table}

\subsection{Abhängigkeit der CCD-Koordinaten in Abhängigkeit der Drivekoordinaten}
Zuletzt wird das Vier-Parameter-Modell benutzt um die CCD-Koordinaten durch die Drive-Koordinaten vorherzusagen. Dazu wird das im Vergleich zu oben invertierte Modell, welches durch die Gleichungen \ref{eq:elD2C4} und \ref{eq:azD2C4} beschrieben wird, verwendet. Für diese Formeln erhält man die in Tabelle \ref{tab:D2C4} Werte und die in Abbildung \ref{img:D2C4} dargestellten Differenzen zwischen Modell und Messwerten.
\begin{figure}[htbp]
\centering
\includegraphics[width=\textwidth]{../341/run341D2C4.png}
\caption{Gezeigt sind die Differenzen zwischen den vom Vier-Parameter-Modell mit $az_0=12,10^{\circ}$, $el_0=-1,19^{\circ}$, $az_1=-0,095^{\circ}$ und $el_1=0,019^{\circ}$ vorhergesagten Drive-Koordinaten und den gemessenen Drive-Koordinaten in Abhängigkeit der CCD-Koordinaten}
\label{img:D2C4}
\end{figure}
\begin{table}[htbp]
\centering
\begin{tabular}{rcl}
\toprule
$el_0$ &=& $(-1,19373\pm 0,9976)^{\circ}$\\
$az_0$ &=& $(12,0957\pm0,04068)^{\circ}$\\
$el_1$ &=& $(0,0192352\pm 1,043)^{\circ}$\\
$az_1$ &=& $(-0,0946443\pm0,1452)^{\circ}$\\
$\sigma$ &=& $0,10^{\circ}$\\
$\rho_{el_0,az_0}$&=& $-0,009202$\\
$\rho_{el_0,el_1}$&=& $-0,0218$\\
$\rho_{az_0,az_1}$&=& $-0,9493$\\
$\rho_{el_0,az_1}$&=& $-0,9954$\\
$\rho_{az_0,el_1}$&=& $-0,03329$\\
$\rho_{el_1,az_1}$&=& $-0,01867$\\
\bottomrule
\end{tabular}
\label{tab:D2C4}
\caption{Die für das Pointingmodell mit vier Parametern bestimmten Werte, wobei die CCD-Koordinaten in Abhängigkeit der Drive-Koordinaten vorhergesagt wurden.}
\end{table}

\section{Vergleich der Parameter}

\section{Diskussion der Korrelation}
Sieht man sich die einzelnen Korrelationen an, so erkennt man, dass sich diese für das jeweils gleiche Modell mit gleich vielen Parametern stark unterscheiden und gerade im Modell mit vier Parametern sehr starke Korrelationen auftreten.
\subsection{Unterschiedliche Korrelationen für Modelle mit gleichen Parametern}
Zunächst fällt auf, dass sich im Modell mit zwei Parametern die Korrelationskoeffizienten stark unterscheiden
\subsection{Hohe Korrelation im Modell mit vier Parametern}

\section{Systematische Abweichungen im entwickelten Pointingmodell}
Betrachtet man die Graphen der Differenzen, so erkennt man, dass noch weitere Effekte das Pointing erschweren. Zunächst wird das weitere Verhalten der Elevation betrachtet. Hier fällt bei der Betrachtung von $\Delta el$ in Abhängigkeit des Azimuts auf, dass der Graph eine Wellenbewegung beschreibt. Aus dem Graph, der die Elevationsabhängigkeit beschreibt, lässt sich hingegen keine keine systematische Abweichung erkennen. Ein solches Verhalten könnte beispielsweise die Verkippung der Turmachse erklären. Durch eine solche Verkippung zeigt die Kamera auf der einen Seite des Teleskops ($az=\theta$) in Richtung einer niedrigeren Elevation und auf der anderen Seite ($az=\theta+180^{\circ}$) in Richtung einer höheren Elevation.\\
Bei der Betrachtung der Graphen der Azimutdifferenzen fällt zunächst auch auf, dass diese sowohl von der Elevation als auch vom Azimut abhängen. Die Abstände $\left| \Delta az \right|$ werden für gleiche für gleiche Azimutwerte mit steigender Elevation größer. Dieser Effekt lässt sich ebenfalls durch die Verkippung der Turmachse erklären erklären, die in diesem Fall einen ähnlichen Effekt hat wie der Winkel zwischen CCD und Drive im oben beschriebenen Modell. Zudem sieht man noch, dass $\Delta az$ vom Azimut abhängt. Für kleine Azimutwerte läuft die Kamera dem Drive ein wenig voraus und für große Azimutwerte läuft sie ein wenig hinterher.\\
Diese Abweichungen sind so klein, dass es hier reicht Pointingmodelle der durch die Gleichungen ((und)) beschriebenen Form zu entwickeln, die nur die Differenzen beschreiben. Ein solches Modell hat Ruslan Konno bereits in seiner Bachelorarbeit für die Single-CCD und die Sky-CCD, die zum damaligen Zeitpunkt noch am Rand des Reflektors befestigt war, entwickelt. 
