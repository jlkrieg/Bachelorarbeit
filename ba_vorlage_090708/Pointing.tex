\chapter{Pointingmodell}
Das Pointing von Teleskopen beschäftigt sich damit, dass das Teleskop so ausgerichtet wird, wie es erwünscht ist. Häufig ist das Problem, dass die eingestellte Position nicht exakt mit der gewünschten Position übereinstimmt. Gründe dafür können Fehler in der Präzision oder auch die Elastizität einzelner Bauteile sein. Da man die aufgenommen Daten mit den Bekannten Postionen am Himmel vergleichen kann, kann man versuchen ein Modell zu finden, welches die Fehler verkleinert oder im Idealfall sogar eliminiert.

\section{Entwicklung von Pointingmodellen}
Da man aus den gewünschten Koordinaten der CCD die Koordinaten des Drives bestimmen lässt, drückt man sie als Funktion voneinander aus.
\begin{equation}
az_D=f_{az}(az_C,el_C)\\
el_D=f_{el}(az_C,el_C)
\label{eq:pointingprinciple}
\end{equation}\\
Diese Funktionen werden so optimiert, dass die Gleichung möglichst gut erfüllt wird.

\section{Vereinfachtes Pointingmodell mit zwei Parametern}
Die Position des Drives und die der CCD werden durch zwei Richtungsvektoren $\vec{r_D}$ und $\vec{r_C}$ beschrieben. Die Startposition des Drives wird in Richtung Norden definiert
\begin{equation}
\vec{r}_D^0=\left(\begin{array}{c} 1 \\ 0 \\ 0 \end{array}\right)
\label{eq:startDrive}
\end{equation}
und kann durch eine Transformation $T$, die aus einer Drehung um die y-Achse $R_y$ und folgender Drehung um die z-Achse in Position besteht, gebracht werden
\begin{equation}
\nonumber
T(az,el)=R_z(el)R_y(az)=
\left(\begin{array}{ccc} \cos(az) & \sin(el) & 0 \\ -\sin(el) & \cos(el) & 0 \\ 0 & 0 & 1\end{array}\right)
\left(\begin{array}{ccc} \cos(el) & 0 &-\sin(el) \\0 & 1 & 0\\ \sin(el) & 0 & \cos(el) \end{array} \right)
\end{equation}\\
T(az,el)=\left(\begin{array}{ccc} \cos(az)\cos(el) & \sin(az) &-\cos(az)\sin(el) \\-\cos(el)\sin(az) & \cos(az) & \sin(az)\sin(el)\\ \sin(el) & 0 & \cos(el) \end{array} \right)
\label{eq:TransformMat}
\end{equation}\\
Da durch diese Transformation der Vektor $\vec{r}_D^0$ in jede Richtung gedreht werden kann laesst sich die Startposition durch den Vektor
\begin{equation}
\vec{r}_C^0=T(az_0,el_0)\vec{r}_D^0=\left(\begin{array}{c} \cos(az_0)\cos(el_0) \\ -\cos(el_0)\sin(az_0) \\ \sin(el_0) \end{array}\right)
\label{eq:startCCD}
\end{equation}
beschreiben. Faehrt man nun das Drivesystem in Position, also wendet die Transformation $T$ auf beide Vektoren an, so erhaelt man folgende Richtungen
\begin{equation}
\vec{r}_D=T(az,el)\vec{r}_D^0=\left(\begin{array}{c} \cos(az)\cos(el) \\ -\cos(el)\sin(az) \\ \sin(el) \end{array}\right)
\label{eq:finDrive}
\end{equation}
und
\begin{equation}
\vec{r}_C=T(az,el)\vec{r}_C^0=\left(\begin{array}{c} \cos(az)\left(\cos(az_0)\cos(el)\cos(el_0)-\sin(el)\sin(el_0)\right)-\cos(el_0)\sin(az)\sin(az_0) \\
\sin(az)\left(\sin(el)\sin(el_0)-\cos(az_0)\cos(el)\cos(el_0)\right)-\cos(az)\cos(el_0)\sin(az_0) \\
\cos(az_0)\cos(el_0)\sin(el)+\cos(el)\sin(el_0) \end{array}\right)
\label{eq:finCCD}
\end{equation}
Aus dem Vektor $\vec{r}_D$ lassen sich die Rechenvorschriften fuer die Ruecktransformation ins Koordinatensystem $(el,az)$ ablesen:
\begin{equation}
el=\arcsin(r_z)
\label{eq:backtrafoEl}
\end{equation}
dsf
\begin{equation}
az=\arctan(r_y,r_x)=\left\{\begin{array}{lr}
\arctan\left(\frac{{r_y}}{{r_x}}\right) & \text{fuer } r_x \textgreater 0  \\
\arctan\left(\frac{{r_y}}{{r_x}}\right)+\pi & \text{fuer } r_x \textless 0,r_y \textgreater 0 \\
\pm \pi \text{fuer }  &  r_x \textless 0,r_y = 0 \\
\arctan\left(\frac{{r_y}}{{r_x}}\right)-\pi & \text{fuer } x \textless 0,r_y \textless 0 \\
+\frac{\pi}{2} & \text{fuer } x = 0,r_y \textgreater 0 \\
-\frac{\pi}{2} & \text{fuer } x = 0,r_y \textless 0 \\
\end{array}
\label{eq:backtrafoAz}
\end{equation}
Als Koordinaten des Drives in Abhaenigkeit der Koordinaten der CCD erhaelt man somit
