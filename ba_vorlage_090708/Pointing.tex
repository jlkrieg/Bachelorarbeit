\chapter{Pointingmodell}
Das Pointing von Teleskopen beschäftigt sich damit, dass das Teleskop so ausgerichtet wird, wie es erwünscht ist. Häufig ist das Problem, dass die eingestellte Position nicht exakt mit der gewünschten Position übereinstimmt. Gründe dafür können Fehler in der Präzision oder auch die Elastizität einzelner Bauteile sein. Da man die aufgenommen Daten mit den Bekannten Postionen am Himmel vergleichen kann, kann man versuchen ein Modell zu finden, welches die Fehler verkleinert oder im Idealfall sogar eliminiert.

\section{Entwicklung von Pointingmodellen}
Da man das Teleskop so ausrichten will, das die man die gewünschte Position vorgibt (Koordinaten der CCD- Index C) und dann die Koordinaten am Drive (Index D) einstellt, sucht man nach Funktionen, die die Koordinaten des Drives in Abhängigkeit von den gewünschten Koordinaten beschreibt. 
\begin{equation}
az_D=f_{az}(az_C,el_C)\\
el_D=f_{el}(az_C,el_C)
\label{eq:pointingprinciple}
\end{equation}\\
az_D=f_{az}(az_C,el_C)\\
el_D=f_{el}(az_C,el_C)
\label{eq:pointingprinciple}
\end{equation}\\
Ziel ist es, die Funktionen so zu optimieren, dass die Differenzen zu den gewünschten Koordinaten verschwinden.
\begin{equation}
\Delta{_az}=f_{az}(az_C,el_C)=0\\
\Delta_{el}=f_{el}(az_C,el_C)=0
\label{eq:pointingZero}
\end{equation}\\

\section{Vereinfachtes Pointingmodell mit zwei Parametern}
Zunächst soll ein Pointingmodell mit zwei Parametern entwickelt werden, bei dem die Kamera in der Parkposition ($el_C=0,az_C=0$) in eine andere Richtung zeigt als das Drivesystem ($el_D=el_0,az_D=az_0$). Die beiden Positionen lassen sich auch durch zwei karthesische Richtungsvektoren $\vec{r_D}$ und $\vec{r_C}$ beschrieben. Somit lässt sich die CCD durch einen Vektor in Richtung Norden
\begin{equation}
\vec{r}_C^0=\left(\begin{array}{c} 1 \\ 0 \\ 0 \end{array}\right)
\label{eq:startCCD}
\end{equation}
beschreiben. Durch eine Drehung um die y-Achse mit dem Winkel $el$ und anschließender Drehung die z-Achse um den Winkel $az$ lässt sich jeder Punkt auf der Einheitskugel erreichen. Die beiden Drehungen lassen sich zu einer Transformation $T(az,el)$ zusammenfassen:
\begin{equation}
\nonumber
T(az,el)=R_z(el)R_y(az)=
\left(\begin{array}{ccc} \cos(az) & \sin(el) & 0 \\ -\sin(el) & \cos(el) & 0 \\ 0 & 0 & 1\end{array}\right)
\left(\begin{array}{ccc} \cos(el) & 0 &-\sin(el) \\0 & 1 & 0\\ \sin(el) & 0 & \cos(el) \end{array} \right)
\end{equation}\\
T(az,el)=\left(\begin{array}{ccc} \cos(az)\cos(el) & \sin(az) &-\cos(az)\sin(el) \\-\cos(el)\sin(az) & \cos(az) & \sin(az)\sin(el)\\ \sin(el) & 0 & \cos(el) \end{array} \right)
\label{eq:TransformMat}
\end{equation}\\
Mit dieser Transformtion lässt sich auch die Startposition des Drives bestimmen:
\begin{equation}
\vec{r}_C^0=T(az_0,el_0)\vec{r}_D^0=\left(\begin{array}{c} \cos(az_0)\cos(el_0) \\ -\cos(el_0)\sin(az_0) \\ \sin(el_0) \end{array}\right)
\label{eq:startDrive}
\end{equation}
Wendet man nun die Gleiche Transformation $T(az,el)$ auf beide Startvektoren an, so erhält man für jedes Koordinatenpaar der CCD die zugehörigen Koordinaten des Drives in Abhängigkeit der Koordinaten der CCD. Für die Richtung der CCD ergibt sich
\begin{equation}
\vec{r}_C=T(az,el)\vec{r}_D^0=\left(\begin{array}{c} \cos(az)\cos(el) \\ -\cos(el)\sin(az) \\ \sin(el) \end{array}\right)
\label{eq:finCCD}
\end{equation}
und für die Richtung des Drives
\begin{equation}
\vec{r}_D=T(az,el)\vec{r}_C^0=\left(\begin{array}{c} \cos(az)\left(\cos(az_0)\cos(el)\cos(el_0)-\sin(el)\sin(el_0)\right)-\cos(el_0)\sin(az)\sin(az_0) \\
\sin(az)\left(\sin(el)\sin(el_0)-\cos(az_0)\cos(el)\cos(el_0)\right)-\cos(az)\cos(el_0)\sin(az_0) \\
\cos(az_0)\cos(el_0)\sin(el)+\cos(el)\sin(el_0) \end{array}\right)
\label{eq:finDrive}
\end{equation}
Aus diesen Richtungsvektoren müssen wieder die ursprünglichen Koordinaten $az$ und $el$ rekonstruiert werden. Die Elevation lässt sich aus der z-Komponente (Höhe) berechnen
\begin{equation}
el=\arcsin(r_z)
\label{eq:backtrafoEl}
\end{equation}
und der Azimutwinkel aus dem Verhältnis von y- zu x-Komponente. Allerdings muss man hierbei beachten, in welchem der 4 Quadranten man sich befindet
\begin{equation}
az=\arctan(r_y,r_x)=\left\{\begin{array}{lr}
\arctan\left(\frac{{r_y}}{{r_x}}\right) & r_x \textgreater 0  \\
\arctan\left(\frac{{r_y}}{{r_x}}\right)+\pi &  r_x \textless 0,r_y \textgreater 0 \\
\pm \pi   &  r_x \textless 0,r_y = 0 \\
\arctan\left(\frac{{r_y}}{{r_x}}\right)-\pi &  x \textless 0,r_y \textless 0 \\
+\frac{\pi}{2} &  x = 0,r_y \textgreater 0 \\
-\frac{\pi}{2} & x = 0,r_y \textless 0 \\
\end{array}
\label{eq:backtrafoAz}
\end{equation}
Die Position der CCD ist so konstruiert, dass ihre Koordinaten mit denen in der Transformationsmatrix übereinstimmen
\begin{equation}
az_C=az\\
el_C=el
\end{equation}
und für die Koordinaten des Drives erhält man
\begin{equation}
az
\end{equation}