\chapter{Pointing von Teleskopen}
Das Pointing von Teleskopen beschäftigt sich damit, dass das Teleskop so ausgerichtet wird, wie es erwünscht ist. Häufig ist das Problem, dass die eingestellte Position nicht exakt mit der gewünschten Position übereinstimmt. Gründe dafür können Fehler in der Präzision oder auch die Elastizität einzelner Bauteile sein. Da man die aufgenommen Daten mit den Bekannten Postionen am Himmel vergleichen kann, kann man versuchen ein Modell zu finden, welches die Fehler verkleinert oder im Idealfall sogar eliminiert.

\section{Koordinatensysteme in der Astronomie}
Als geeignetes Koordinatensystem für den Betrieb eines Teleskops erweist sich ein mit zwei Winkeln zu beschreibendes System, das den Kugelkoordinaten ähnelt. Der Azemutwinkel behält seinen Namen und zeigt in der Regel bei $0^\circ$ in Richtung Norden. Der Polarwinkel behält ebenfalls seine Bedeutung und wird Elevation genannt.

\section{Äquatoriales Koordinatensystem}
Um die Position von Sternen eindeutig zu identifizieren benötigt man ein weiteres Koordinatensytem, das von der Position des Frühlingspunktes abhängig ist. Von diesem Punkt ausgehend kann jeder Punkt durch die beiden Winkel Rektaszension und Deklination beschrieben werden.

\section{irgendwas}
