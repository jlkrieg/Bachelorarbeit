\chapter{Bildanalyse}
Zu Beginn war der MST Protoyp in Adlershof noch nicht mit einem Cherenkovdetektor ausgestattet, sondern nur mit einfachen CCDs. Mit diesen wurde die Helligkeit des Nachthimmels beobachtet.

\section{CCD Kameras}

\section{Verwendete Kamera}
Das MST ist mit verschiedenen Kameras ausgestattet, wobei nur Bilder der sogenannten Sky-CCD verwendet wurden. Die Sky-CCD ist eine ist eine Kamera des Typs Prosilica GC 1350 mit folgenden technischen Daten.

Die Bilder wurden mit mit drei verschieden Belichtungszeiten (1s, 10s und 20s) und vier verschieden gain-Verstärkungsstufen (0dB, 7dB,14dB und 21dB) aufgenommen. Die Bilder wurden in Schwarz-Weiß mit einer Farbtiefe von 8Bit aufgenommen, das heißt jedem Pixel wird ein Wert von 0 bis 255 zugewiesen, wobei der Wert 255 der maximalen Helligkeit entspricht.

\section{Helligkeit der Bilder}
Um die Helligkeit der Bilder zu bestimmen wurde auf das arithmetische Mittel verzichtet, da dieses durch den Einfluss heißer Pixel in Richtung zu hoher Helligkeit verschoben wird. Heiße Pixel sind Pixel, die nicht ordnungsgemäß funktionieren und nicht proportional auf das einfallende Licht reagieren, sondern schneller hell werden. Gerade bei längeren Belichtungszeiten kommt es so vor, dass diese Pixel auch bei eher dunklen Bildern des Nachthimmels den maximalen Helligkeitswert annehmen. Um diesen Effekt zu minimieren, wurde jeweils der Median der Verteilung berechnet. Da die Helligkeit der Pixel der Digitalkamera nur ganzzahlige Werte annehmen kann, aber gerade im dunklen Bereich eine präzisere Helligkeit erreicht werden soll, wurde die Verteilung innerhalb eines Bins als kontinuierlich. Zudem wurde noch die Breite der Verteilung berechnet. Dazu wurde der Bereich einer Standardabweichung also 37, \% links und rechts des zuvor berechneten Medians gewählt.

Zur Analyse des Zusammenhangs der Belichtungszeit bzw des gains auf die Helligkeit der Bilder wurde der Datensatz "run 199" verwendet, der am von bis aufgenommen wurde. Für jedes einzelne Bild wurde die Belichtungszeit und der gain sowie wie oben beschrieben der Median der Helligkeitsverteilung sowie deren Breite bestimmt

\section{Korrelation der Werte}
Im folgenden soll untersucht werden, wie sich Helligkeit und Breite in Abhängigkeit der Belichtungszeit und des gains verhalten.

\subsection{Abhängigkeit von der Belichtungszeit}
Eine längere Belichtungszeit bedeutet, dass die Blende der Kamera länger geöffnet bleibt. Daraus folgt die Erwartung, dass die Anzahl der detektierten Photonen proportional steigt und somit auch der Median der Helligkeitsverteilungen.

\subsection{Abhängigkeit vom gain}

\section{Fazit}