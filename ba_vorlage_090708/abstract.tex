%-englische-Zusammenfassung---------------------------------------
%
%\selectlanguage{english}
%
%\begin{abstract}
%\setcounter{page}{2} % Nach Bedarf anpassen!
%Here is the english abstract.\\
%% hier werden die englische Schlagw�rter aus Metadaten �bernommen
%\dckeywordsen				
%\end{abstract}

%-deutsche Zusammenfassung----------------------------------------

%\selectlanguage{german}

\begin{abstract}
%\setcounter{page}{3} % Nach Bedarf anpassen!
%Das CTA ist ein internationales Projekt, dass die aktuelle Generation an Tscherenkow-Teleskopen ersetzen soll. F\"ur eines der Teleskope in Berlin ein Prototyp errichtet, um den Aufbau und die Funktion zu testen. Um das Pointing zu testen befinden sich im Zentrum des Reflektors drei CCD-Kameras, von denen eine schr�g montiert ist. F�r diese wird ein Modell entwickelt, das die durch den festen Winkel entstehenden Abweichungen korrigiert.\\
Das CTA ist ein internationales Projekt, das die aktuelle Generation an Observatorien zur bodengest\"utzten Detektion hochenergetischer Gammastrahlung abl\"osen soll. F\"ur einen der geplanten Teleskoptypen -das Medium-Sized-Telescope- wurde in Berlin ein Prototyp errichtet, um beispielsweise die korrekte Ausrichtung zu testen. Dazu werden drei CCD-Kameras verwendet, wovon eine einen gro\ss en Winkel zur optischen Achse des Teleskop hat. Die Beziehung der Koordinaten von Teleskop und CCD-Kamera lassen sich durch ein Modell mit zwei Parametern beschreiben.\\
% hier werden die deutsche Schlagw�rter aus Metadaten �bernommen
%\dckeywordsde
\end{abstract}

