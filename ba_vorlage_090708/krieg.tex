%%%%%%%%%%%%%%%%%%%%%%%%%%%%%%%%%%%%%%%%%%%%%%%%%%%%%%%%%%%%%%%%%%%
%
% Dissertationen mit LaTeX auf dem edoc-Server
%
% Humboldt-Universitaet zu Berlin
% Computer- und Medienservice
% Arbeitsgruppe Elektronisches Publizieren
% Bezug der Vorlage und der Richtlinien:
%     http://edoc.hu-berlin.de/e_autoren/latex/
%
% Kontakt:
%     E-Mail:
%                   edoc-latex@rz.hu-berlin.de
%     Telefon:              siehe
%     http://edoc.hu-berlin.de/e_autoren/latex/kontakt.php  %
%
%%%%%%%%%%%%%%%%%%%%%%%%%%%%%%%%%%%%%%%%%%%%%%%%%%%%%%%%%%%%%%%%%%%%%%%
%
% Das folgende Template muss f�r die Publikation von digitalen        %
% Dissertationen in LaTeX an der Humboldt-Universit�t benutzt werden. %
%
% Aendern Sie den Name dieser Datei auf ihren_nachname.tex.           %
%
% Die mit einem Stern (*) gekennzeichnete Teile sind optional;        %
% falls Sie sie nicht verwenden m�chten, sind entsprechende Zeile       %
% zu entfernen.
%
%%%%%%%%%%%%%%%%%%%%%%%%%%%%%%%%%%%%%%%%%%%%%%%%%%%%%%%%%%%%%%%%%%%%%%%

% \listfiles    % Erstellt eine Liste von allen benutzten Dateien
% zusammen mit ihrer Versionen
% und ggf. einer kurzen Beschreibung
% Sie wird in die log-Datei geschrieben

\documentclass[12pt,a4paper% die Verwendung von DIN-A4-Format ist pflicht!
]{report}
\usepackage[top=2cm, bottom=2cm, left=2cm, right=2cm]{geometry}

%notwendige Pakete
\usepackage[ngerman, english]{babel}    % mehrsprachiger Textsatz
% babel: letzte Sprache in Optionen zeigt die Sprache des Dokumentes
% und kann durch den Befehl \selectlanguage{} geaendert werden
% Passen Sie die Optionen des babel-Paketes nach Bedarf an!
\usepackage[ansinew]{inputenc}       % Eingabekodierung Parameter latin1 darf ge�ndert werden
\usepackage[T1]{fontenc}                % Schriftenkodierung
\usepackage{graphicx}                       % zum Einbinden von Grafiken
\usepackage{lmodern}                        % Ersatz fuer Computer Modern-Schriften
                                                                % zum besseren Aussehen am Bildschirm
                                                                
\include{metadata}                          % Bitte ALLE Angaben erf�llen!
\include{compatibility}

%-Eigene Trennregeln*---------------------------------------------

% \include{hyphenations}

%-zusaetzliche Kommandos*-----------------------------------------

\renewcommand{\chaptername}{}

\renewcommand{\thechapter}{}
%\include{command}

%-Dokument--------------------------------------------------------

\begin{document}

% Es muss zitiert werden k�nnen! Im Vorspann roemisch,
% Im Hauptteil benutzt man arabische Nummerierung.
\pagenumbering{roman}

%-Titelblatt------------------------------------------------------

\include{titlepage}                         % Bitte KEINE �nderungen vornehmen!
\maketitle


%-Zusammenfassung / Abstract*-------------------------------------

%%-englische-Zusammenfassung---------------------------------------
%
%\selectlanguage{english}
%
%\begin{abstract}
%\setcounter{page}{2} % Nach Bedarf anpassen!
%Here is the english abstract.\\
%% hier werden die englische Schlagw�rter aus Metadaten �bernommen
%\dckeywordsen				
%\end{abstract}

%-deutsche Zusammenfassung----------------------------------------

%\selectlanguage{german}

\begin{abstract}
%\setcounter{page}{3} % Nach Bedarf anpassen!
Das CTA ist ein internationales Projekt, dass die aktuelle Generation an Tscherenkow-Teleskopen ersetzen soll. F\"ur eines der Teleskope in Berlin ein Prototyp errichtet, um den Aufbau und die Funktion zu testen. Um das Pointing zu testen befinden sich im Zentrum des Reflektors drei CCD-Kameras, von denen eine schr�g montiert ist. F�r diese wird ein Modell entwickelt, das die durch den festen Winkel entstehenden Abweichungen korrigiert.\\
% hier werden die deutsche Schlagw�rter aus Metadaten �bernommen
\dckeywordsde
\end{abstract}


%\selectlanguage{english}               % Bitte an die Sprache denken!!!
%\setcounter{page}{4}                   %   Bitte an die Seitenzahl denken!!!

%-Widmung*--------------------------------------------------------

\include{dedication}

%-Inhaltsverzeichnis----------------------------------------------

\tableofcontents
\pagebreak

\listoffigures
\pagebreak

\listoftables

%-Hauptteil-------------------------------------------------------

\pagenumbering{arabic}
\pagestyle{myheadings}                  % bzw. ist fancyhdr zu benutzten

%-Kapitel---------------------------------------------------------

% part ist optional, bitte ggf. l�schen
% \part{Teil1}

%\chapter{Erstes Kapitel}
\section{Erster Abschnitt Kapitel 1}
\subsection{Erster Unterabschnitt }
Hier soll jetzt mal zitiert werden. \cite{av:1a}
\texorpdfstring{tex}{pdf}
% \include{chapter02}
\chapter{gamma-Astronomie}
Die Astronomie ist die Wissenschaft des Universums und beschreibt die Bewegung und Eigenschaften von Himmelskörpern wie Planeten oder Galaxien, interstellarer Materie und Strahlung. Betrachtete man frueher nur Licht im optisch sichtbaren Bereich, so sind im 20. Jahrhundert einige zusatliche Quellen dazugekommmen. Dazu zaehlen die von Viktor HESS durch Ballonversuche entdeckte kosmische Strahlung, die Roentgen-/bzw die Gammastrahlung sowie die Neutrinoastronomie. Die Gammaastronomie beschaftigt sich mit Photonen im Bereich von bis . Photonen haben den Vorteil, dass sie nicht wie geladene Teilchen durch elektromagnetische Felder abgelenkt werden und somit ihre Quelle leichter detektiert werden kann. Zudem sind sie auch noch deutlich leichter zu detektieren sind als Neutrinos. Da die Energie dieser Photonen so hoch ist, koennen sie nicht thermischen Ursprungs sein sondern kommen aus anderen Quellen, deren Untersuchung das Ziel der Hochenergie-Gamma-Astronomie ist.

%cite Design concept

\section{Entstehung hochenergetischer Strahlung}
\subsection{inverser Comptoneffekt}
Durch den Comptoneffekt koennen hochenergetische Photonen einen Teil ihres Impulses und Energie an ein freies Elektron uebergeben. Dieser Prozess kann auch invers ablaufen und somit kann ein niederenergetisches Photon, zum Beispiel aus dem kosmischen Mikrowellenhintergrund (E), durch einen Stoss mit einem Elektron eine hohe Energie bekommen.
\subsection{Zerfall von schweren Teilchen}
Zerfallen schwerere Teilchen in Photonen, so wird die Ruheenergie dieses Teilchens in kinetische Energie der Photonen umgewandelt. Ein Beispiel hierfuer ist der Zerfall des neutralen Pions, die haufig bei der Kollision von Atomkernen entstehen. Das Pion hat eine Ruhemasse von 135MeV 
%cite pdg
und zerfaellt fast ausschliesslich in zwei Photonen, die dann eine Energie von ungefaehr 68 MeV haben.
\subsection{Materie-Antimaterie-Annihilation}
Bei der Kollision von Materie mit Antimaterie vernichten sich die beiden Teilchen und es entstehen Neue. Dies koennen Photonen sein oder Teichen, die wiederum in Photonen zerfallen. Ein prominetes Beispiel hierfuer ist die Elektron-Positron-Annihilation. Besitzen die beiden Teilchen keine kinetische Energie, so zerfallen sie in zwei Photonen mit der Energie E=511keV.
\subsection{Bremstrahlung}
Durchfliegen hochenergetische Teilchen Materie, so kann es vorkommen, dass diese eng an den Atomen vorbeifliegen und abgelenkt werden. Durch diese Ablenkung werden Photonen abgestrahlt.
\subsection{unbekannte Effekte}
Hochenergetische Photonen koennen auch nach Prinzipien erzeugt werden, die man heute noch nicht versteht. So koennte es moeglich sein, dass hochenergetische Photonen durch den Zerfall von Partikeln der dunklen Materie stammen. Die Supersymmetrie sagt zum Beispiel den Zerfall von schweren WIMPS in Photonen vorraus. Durch Detektion solcher Ereignisse liesse sich auf neue Physik schliessen.

\section{Quellen hochenergetischer Strahlung}
Ziel VHE-Astronomie ist es die Quellen hochenergetischer Gammastrahlung zu erforschen. Folgende Quellen sind bekannt:
\subsection{Supernova Ueberreste}
\subsection{Pulsare}
\subsection{Quasare}
\subsection{Stelare}
\subsection{Aktive Galaxien}
\subsection{Binaere Systeme}
\subsection{Gamma Ray Bursts}


\section{Detektion von Strahlung}
Prinzipiell laesst sich zwischen bodengestuetzter und satellitengestuetzter Gammaastronomie unterscheiden. Durch den Einsatz von Satelliten vermeidet man den stoerenden Einfluss der Erdatmosphaere, muss dafuer Abstriche in der Groesse der Detektoren machen und mit hohen Kosten kalkulieren. Hier soll sich nur mit der bodengestuetzten Variante beschaeftigt werden.

\subsection{Luftschauer}
Treten hochenergetische Photonen in die Materie ein, so wechselwirken sie mit dieser ueber Paarbildung. Das entstehende Elektron bzw Positron verliert daraufhin Energie durch Bremstrahlung, worauf die entstehenden Photonen wieder durch Paarbildung wechselwirken koennen. Somit steigt die Anzahl der Teilchen exponentiell an und die durchschnittliche Energie nimmt exponentiell ab, bis die Teilchen ioniserend sind und der Schauer verschwindet. Die entstehenden Teilchen lassen sich nicht direkt nachweisen, da der Schauer bereits in einer Hoehe von ca 10km verschwindet. %bild zur veranschaulichung

\subsection{Cherenkov Strahlung}
Cherenkov Strahlung tritt auf, wenn geladene sich Teilchen in Materie schneller als Photonen bewegen und lässt sich analog zum Überschallknall erklären. Das geladene Teilchen polarisiert auf seiner Trajektorie die einzelnen Atome, die somit Licht sphaerisch abstrahlen. Da sich das Teilchen allerdings schneller als das Licht bewegt, entsteht ein Kegel konstruktiver Interferenz. Somit entsteht ein Lichtblitz, der sich kegelfoermig mit dem Oeffnungswinkel %bild zur interferenz
\begin{align}
\theta = \arccos\left(\frac{1}{\beta n}
\end{align}
ausbreitet.

\subsection{bodengestuezte Detektion mit Teleskopen}
Da aufgrund der Atmosphaere weder das primaere Photon noch die Teilchen des Luftschauers detektiert werden koennen, versucht man die Cherenkovstrahlung, die durch den Luftschauer entsteht zu detektieren. Dazu muss eine grosse Flaeche abgedeckt werden, da selbst bei vertikaler Einstrahlung der Schauer einen Durchmesser von ca 250m haben kann %design concept

\chapter{Das Cherenkov Telescope Array}
Mit dem Bau des Cherenkov Telescope Arrays (CTA) werden verschiedene Ziele verfolgt:
\begin{itemize}

\item Verbesserung des Senstitiviteatslevels um eine Grossenordnung auf 1TeV
\item Erhoehung der Detektionsflaeche/Photonenrate->Zugang zu kurzeitigen Ereignissen
\item Erhoehung der Winkelaufloesung/des Sichtfeldes
\item Energieabdeckung von 20GeV bis 300TeV
\item Verbesserung des Vermessungsfaehigkeit, Ueberwachungsfaehigkeit und Flexibilitaet->gleichzeitige Beobachtung von Objekten in verschiedenen Feldern
\item Datenproduktion und Tools auf fuer nicht Experten
\item Abdeckung des gesamten Himmels (nord+sued)
\end{itemize}

\section{Design-Konzept}
Um sowohl die suedliche als auch die noerdliche Hemnisphaere abzudecken, wird das CTA in der Atacamawueste in Chile und auf der zu Spanien gehoerenden Insel La Palma errichtet.


\section{Prototyp in Adlershof}
In Adlershof wurde 2012 vom DESY ein Prototyp des MSTs errichtet um den mechanischen Aufbau zu testen, Pointingmodelle zu entwickeln und um die einzelnen Spiegel zu testen und auszurichten.

\subsection{Kameras des MST}
Der Prototyp des MST besitzt drei Kameras in der Mitte des Reflektors. Die Sky-CCD, für die im hier folgenden ein Pointingmodell entwickelt wird, ist schräg montiert, sodass sie am Detektorarm vorbei guckt um Bilder des Nachthimmels aufzunehmen. Aus diesen Bildern lassen sich mithilfe der Astrometry-Software die Koordinaten der Kamera bestimmen, die als die wahren Koordinaten angenommen werden.

\subsection{Koordinatens des MST}
Als geeignetes Koordinatensystem für den Betrieb eines Teleskops erweist sich ein mit zwei Winkeln zu beschreibendes System, das den Kugelkoordinaten ähnelt. Der Azemutwinkel behält seinen Namen und zeigt in der Regel bei $0^\circ$ in Richtung Norden. Der Polarwinkel behält ebenfalls seine Bedeutung und wird Elevation genannt.
\chapter{Bildanalyse}
Zu Beginn war der MST Protoyp in Adlershof noch nicht mit einem Cherenkovdetektor ausgestattet, sondern nur mit einfachen CCDs. Mit diesen wurde die Helligkeit des Nachthimmels beobachtet.

\section{CCD Kameras}

\section{Verwendete Kamera}
Das MST ist mit verschiedenen Kameras ausgestattet, wobei nur Bilder der sogenannten Sky-CCD verwendet wurden. Die Sky-CCD ist eine ist eine Kamera des Typs Prosilica GC 1350 mit folgenden technischen Daten.

Die Bilder wurden mit mit drei verschieden Belichtungszeiten (1s, 10s und 20s) und vier verschieden gain-Verstärkungsstufen (0dB, 7dB,14dB und 21dB) aufgenommen. Die Bilder wurden in Schwarz-Weiß mit einer Farbtiefe von 8Bit aufgenommen, das heißt jedem Pixel wird ein Wert von 0 bis 255 zugewiesen, wobei der Wert 255 der maximalen Helligkeit entspricht.

\section{Helligkeit der Bilder}
Um die Helligkeit der Bilder zu bestimmen wurde auf das arithmetische Mittel verzichtet, da dieses durch den Einfluss heißer Pixel in Richtung zu hoher Helligkeit verschoben wird. Heiße Pixel sind Pixel, die nicht ordnungsgemäß funktionieren und nicht proportional auf das einfallende Licht reagieren, sondern schneller hell werden. Gerade bei längeren Belichtungszeiten kommt es so vor, dass diese Pixel auch bei eher dunklen Bildern des Nachthimmels den maximalen Helligkeitswert annehmen. Um diesen Effekt zu minimieren, wurde jeweils der Median der Verteilung berechnet. Da die Helligkeit der Pixel der Digitalkamera nur ganzzahlige Werte annehmen kann, aber gerade im dunklen Bereich eine präzisere Helligkeit erreicht werden soll, wurde die Verteilung innerhalb eines Bins als kontinuierlich. Zudem wurde noch die Breite der Verteilung berechnet. Dazu wurde der Bereich einer Standardabweichung also 37, \% links und rechts des zuvor berechneten Medians gewählt.

Zur Analyse des Zusammenhangs der Belichtungszeit bzw des gains auf die Helligkeit der Bilder wurde der Datensatz "run 199" verwendet, der am von bis aufgenommen wurde. Für jedes einzelne Bild wurde die Belichtungszeit und der gain sowie wie oben beschrieben der Median der Helligkeitsverteilung sowie deren Breite bestimmt

\section{Korrelation der Werte}
Im folgenden soll untersucht werden, wie sich Helligkeit und Breite in Abhängigkeit der Belichtungszeit und des gains verhalten.

\subsection{Abhängigkeit von der Belichtungszeit}
Eine längere Belichtungszeit bedeutet, dass die Blende der Kamera länger geöffnet bleibt. Daraus folgt die Erwartung, dass die Anzahl der detektierten Photonen proportional steigt und somit auch der Median der Helligkeitsverteilungen.

\subsection{Abhängigkeit vom gain}

\section{Fazit}
\chapter{Pointingmodell für eine große Abweichung zur optischen Achse}
\label{ch:pointing}
Das Pointing von Teleskopen beschäftigt sich damit, dass das Teleskop so ausgerichtet wird, wie es erwünscht ist. Häufig ist das Problem, dass die eingestellte Position nicht exakt mit der gewünschten Position übereinstimmt. Gründe dafür können Fehler in der Präzision oder auch die Elastizität einzelner Bauteile sein. Im Fall der Sky-CCD des MST-Prototypen liegt es wie in Abschnitt \ref{se:cameras} beschrieben an der zentralen Befestigung die zu einem größeren Winkel zur optischen Achse führt. Da man die aufgenommen Daten mit den Bekannten Postionen am Himmel vergleichen kann, kann man versuchen ein Modell zu finden, welches die Fehler verkleinert oder im Idealfall sogar eliminiert.

\section{Koordinaten}
Das MST benutzt ein Koordinatensystem aus zwei Winkeln, welches den Kugelkoordinaten ähnelt. Der Azimutwinkel ($az$) beschreibt die Auslenkung in der Ebenen und läuft von $-180^{\circ}$ bis $180^{\circ}$, wobei es für $az=0^{\circ}$ in Richtung Norden ausgerichtet ist. Der Elevationswinkel $el$ läuft von $0^{\circ}$ bis $90^{\circ}$ wobei $el=90^{\circ}$ dem Zenit entspricht. Hier wird zudem die Konvention benutzt, dass Nordrichtung der x-Richtung, die Westrichtung der y-Richtung und die Zenitrichtung der z-Richtung entspricht.
\begin{figure}[htbp]
\centering
\includegraphics[width=0.7\textwidth]{Images/coordinates.png}
\caption{Die verwendeten Koordinaten}
\label{img:coordinates}
\end{figure}

\section{Entwicklung von Pointingmodellen}
Um mit einem Teleskop möglichst gute Resultate zu erzielen muss man die gewünschte Position am Himmel so genau wie möglich einstellen. Da die eingestellten Koordinaten allerdings nicht exakt mit den tatsächlichen Koordinaten übereinstimmen, verwendet man Pointingmodelle um diese Differenzen zu minimieren. Da Teleskope in der Regel so konstruiert sind, dass es möglichst geringe Abweichungen zwischen eingestellten und tatsächlichen Positionen existieren, reicht es aus, sich die Differenzen anzgucken. Diese Differenzen lassen sich als dann als Funktion der teleskopspezifischen Paramater $\vec{q}$ entwickeln.
\begin{equation}
az_D=az_c+\Delta_{az}=az_C+\tilde{f}_{az}\left(az_c,el_c,\vec{q}\right)
\end{equation}
\begin{equation}
el_D=el_c+\Delta_{el}=el_C+\tilde{f}_{el}\left(az_c,el_c,\vec{q}\right)
\label{eq:pointingZero}
\end{equation}
Als Beispiel wird hier die Verkippung der Teleskopachse, die Ruslan Konno in seiner Bachelorarbeit beschrieben hat \citep{Ruslan}, in vereinfachter Form disktutiert. Dazu wird nur die Elevation betrachtet, die sich aus der z-Komponente berechnen lässt
\begin{equation}
sin(el)=z.
\end{equation}
Für ein kleines $\Delta_{el}$ lässt sich diese Formel gut durch eine Taylorentwicklung bis zum linearen Glied beschreiben.
\begin{equation}
\sin(el+\Delta_{el})=\sin(el)\cos(\Delta_{el})+\cos(el)sin(\Delta_{el})\approx \sin(el)+\cos(el)\Delta_{el}
\end{equation}
Dreht man nun einen beliebigen Vektor mit einem kleinen Winkel um die x-Achse
\begin{equation}
\left(\begin{array}{c}
x\prime\\y\prime\\z\prime
\end{array}\right)=\left(\begin{array}{ccc}
1 & 0 & -\phi_X\\0 & 1 & 0 \\\phi_X & 0 & 1
\end{array}\right)\left(\begin{array}{c}
x\\y\\z
\end{array}\right)=\left(\begin{array}{c}
x-\phi_Xz\\y\\\phi_X+z
\end{array}\right),
\end{equation}
lässt sich aus der z-Komponente die Abweichung $\Delta_{el}$ berechnen.
\begin{equation}
\sin(el)+\cos(el)\Delta_{el}=z\prime=\phi_Xx+z
\end{equation}
mit Gleichung xx ergibt sich somit 
\begin{equation}
\Delta_{el}=\frac{\phi_Xx}{\cos(el)}
\end{equation}
als Abweichung. Aus dieser Formel kann man sehen, dass diese Approximation nur für kleine Unterschiede, wie in diesem Fall eine kleine Drehung um die x-Achse funktioniert.
%Da man das Teleskop so ausrichten will, dass man die gewünschte Position vorgibt (Koordinaten der CCD- Index C) und dann die Koordinaten am Drive (Index D) einstellt, sucht man nach Funktionen, die die Koordinaten des Drives in Abhängigkeit von den gewünschten Koordinaten beschreibt. \\
%In der Regel haben Teleskope so geringe Abweichungen zwischen eingestellter und tatsächlicher Position, dass sich diese Abweichungen als Funktionen schreiben lassen, die in den relevanten Parametern entwickelt werden können.
%\begin{equation}
%az_D=az_c+\Delta_{az}=az_C+\tilde{f}_{az}\left(az_c,el_c,\vec{q}\right)
%\end{equation}
%\begin{equation}
%el_D=el_c+\Delta_{el}=el_C+\tilde{f}_{el}\left(az_c,el_c,\vec{q}\right)
%\label{eq:pointingZero}
%\end{equation}
Da die Sky-CCD des MST-Prototyps einen größeren Winkel ($\mathcal{O}\left(10^{\circ}\right)$) zur optischen Achse hat, lässt sich das Entwicklungsverfahren hier nicht anwenden. Stattdessen wird versucht, die einzustellende Position anhand geometrischer Überlegungen in Abhängigkeit der wahren Position vorherzusagen. Man versucht also Funktionen zu finden, die nur von den wahren Koordinaten und einem teleskopsspezifischen Parametersatz abhängen.
\begin{equation} 
az_D=f_{az}(az_C,el_C)
\end{equation}
\begin{equation}
el_D=f_{el}(az_C,el_C)
\label{eq:pointingprinciple}
\end{equation}

\section{Pointingmodell mit zwei Parametern}
\subsection{Vorhersage der CCD-Koordinaten in Abhängigkeit der Drive-Koordinaten}
Zunächst soll ein Pointingmodell mit zwei Parametern entwickelt werden, bei dem das Drivesystem in der Parkposition ($el_D=0,az_D=0$) in eine andere Richtung zeigt als die CCD-Kamera ($el_C=el_0,az_C=az_0$). Die beiden Positionen lassen sich auch durch zwei kartesische Richtungsvektoren $\vec{r_D}$ und $\vec{r_C}$ beschrieben. Ausgehend von dieser Startposition wird das Teleskop beziehungsweise die Vektoren $\vec{r_D}$ und $\vec{r_C}$ durch orthogonale Transformationen in die gewünschte Position gebracht wird. Als Parkpostion für das Drive-System erhält man mit den oben genannten Bedingung folgenden karthesischen Vektor:
\begin{equation}
\vec{r}_D^0=\left(\begin{array}{c} 1 \\ 0 \\ 0 \end{array}\right).
\label{eq:startDrive}
\end{equation}
Durch eine Drehung um die y-Achse mit dem Winkel $el$ und anschließender Drehung die z-Achse um den Winkel $az$ lässt sich aus dieser Startposition jeder Punkt auf der Einheitskugel erreichen. Die beiden Drehungen lassen sich zu einer Transformation $T(az,el)$ zusammenfassen:
\begin{equation}
T(az,el)=R_z(az)R_y(el)=
\left(\begin{array}{ccc} \cos(az) & \sin(az) & 0 \\ -\sin(az) & \cos(az) & 0 \\ 0 & 0 & 1\end{array}\right)
\left(\begin{array}{ccc} \cos(el) & 0 &-\sin(el) \\0 & 1 & 0\\ \sin(el) & 0 & \cos(el) \end{array} \right)
\end{equation}\\
\begin{equation}
T(az,el)=\left(\begin{array}{ccc} \cos(az)\cos(el) & \sin(az) &-\cos(az)\sin(el) \\-\cos(el)\sin(az) & \cos(az) & \sin(az)\sin(el)\\ \sin(el) & 0 & \cos(el) \end{array} \right).
\label{eq:TransformMat}
\end{equation}\\
Unter der Annahme, dass die Kamera von vornherein in eine andere Richtung als das Drive-System zeigt, lässt sich mit dieser Transformtion $T(az0,el0)$ aus der Startpostion des Drives die Startposition der Kamera bestimmen:
\begin{equation}
\vec{r}_C^0=T(az_0,el_0)\vec{r}_D^0=\left(\begin{array}{c} \cos(az_0)\cos(el_0) \\ -\cos(el_0)\sin(az_0) \\ \sin(el_0) \end{array}\right).
\label{eq:startCCD}
\end{equation}
Da hier angenommen wird, dass der Winkel zwischen der optischen Achse des Teleskops und der Kamera fest ist, sind $az_0$ und $el_0$ Konstanten. Wendet man nun die gleiche Transformation $T(az_D,el_D)$ mit den Koordinaten des Teleskops als Argumente auf beide Startvektoren an, so erhält man für jedes Koordinatenpaar des Drives die zugehörigen Koordinaten der CCD in Abhängigkeit der Koordinaten des Drives. Für dessen Richtung ergibt sich
\begin{equation}
\vec{r}_D=T(az_D,el_D)\vec{r}_D^0=\left(\begin{array}{c} \cos(az)\cos(el) \\ -\cos(el)\sin(az) \\ \sin(el) \end{array}\right)
\label{eq:finDrive}
\end{equation}
und für die Richtung der CCD
\begin{equation}
\vec{r}_C=T(az,el)\vec{r}_C^0=\left(\begin{array}{c} \cos(az)\left(\cos(az_0)\cos(el)\cos(el_0)-\sin(el)\sin(el_0)\right)-\cos(el_0)\sin(az)\sin(az_0) \\
\sin(az)\left(\sin(el)\sin(el_0)-\cos(az_0)\cos(el)\cos(el_0)\right)-\cos(az)\cos(el_0)\sin(az_0) \\
\cos(az_0)\cos(el_0)\sin(el)+\cos(el)\sin(el_0) \end{array}\right).
\label{eq:finCCD}
\end{equation}
Zur Untersuchung der Azimutabhängigkeit werden aus Gründen der Übersichtlichkeit folgende Terme substituiert
\begin{equation}
\cos\left(el_D\right)\cos\left(az_0\right)\cos\left(el_0\right)-\sin\left(el_D\right)\sin\left(el_0\right)=a
\end{equation}
\begin{equation}
\sin\left(az_0\right)\cos\left(el_D\right)=b.
\end{equation}
Gleichung \ref{eq:finCCD} lässt sich nun als
\begin{equation}
\vec{r}_C=T(az,el)\vec{r}_C^0=\left(\begin{array}{c} 
a\cos(az)-b\sin(az)\\
-a\sin(az)-b\cos(az)\\
\cos(az_0)\cos(el_0)\sin(el)+\cos(el)\sin(el_0) \end{array}\right)
\label{eq:finCCDab}
\end{equation}
schreiben. Aus diesen Richtungsvektoren müssen wieder die ursprünglichen Koordinaten $az$ und $el$ rekonstruiert werden. Die Elevation lässt sich aus der z-Komponente (Höhe) berechnen
\begin{equation}
el=\arcsin(r_z)
\label{eq:backtrafoEl}
\end{equation}
und der Azimutwinkel aus dem Verhältnis von y- zu x-Komponente. Da der Tangens dieses Verhältnis dem Azimuthwinkel mit einem Wertebereich von $-180^{\circ}$ bis $180^\circ$ entspricht muss man als Umkehrfunktion den erweiterten $\arctan$ verwenden:
\begin{equation}
\arctan 2(r_y,r_x)=\left\{\begin{array}{lr}
\arctan\left(\frac{{r_y}}{{r_x}}\right) & r_x \textgreater 0  \\
\arctan\left(\frac{{r_y}}{{r_x}}\right)+\pi &  r_x \textless 0,r_y \textgreater 0 \\
\pm \pi   &  r_x \textless 0,r_y = 0 \\
\arctan\left(\frac{{r_y}}{{r_x}}\right)-\pi &  x \textless 0,r_y \textless 0 \\
+\frac{\pi}{2} &  x = 0,r_y \textgreater 0 \\
-\frac{\pi}{2} & x = 0,r_y \textless 0 \\
\end{array}\right
\end{equation}

Das Koordinatensystem wurde so gewählt, dass der Azimutwinkel wie in Abbildung \ref{img:coordinates} zu sehen in Richtung Osten also der negativen y-Richtung geht. $az$ lässt sich also durch
\begin{equation}
az=\arctan 2(-r_y,r_x)
\label{eq:backtrafoAz}
\end{equation}
berechnen. Somit lassen sich mit \ref{eq:backtrafoEl} beziehungsweise mit \ref{eq:backtrafoAz} in Verbindung mit \ref{eq:finCCD} die CCD-Koordinaten in Abhängigkeit der Drive-Koordinaten bestimmen.
\begin{equation}
el_C=\arcsin\left(\cos(az_0)\cos(el_0)\sin(el_D)+\sin(el_0)\cos(el_D)\right)
\label{eq:elD2C}
\end{equation}
%\begin{equation}
%az_C=\arctan2(
%\sin(az_D)(\cos(el_D)\cos(az_0)\cos(el0)-\sin(el_D)\sin(el_0))-\cos(az_D)\sin(az_0)\cos(el0),
%\cos(az_D)(\cos(el_D)\cos(az_0)\cos(el0)-\sin(el_D)\sin(el_0))+\sin(az_D)\sin(az_0)\cos(el0))
%\label{eq:azD2C}
%\end{equation}
\begin{equation}
az_C=\arctan2(
a\sin(az_D)+b\cos(az_D),a\cos(az_D)-b\sin(az_D))
\label{eq:azD2C}
\end{equation}

\subsection{Vorhersage der Drive-Koordinaten in Abhängigkeit der CCD-Koordinaten}
Um die Abhängigkeiten der Drive-Koordinaten von den CCD-Koordinaten zu erhalten, muss das aus \ref{eq:elD2C} und \ref{eq:azD2C} bestehende Gleichungssystem nach $el_D$ und $az_D$ aufgelöst werden. Da die Formel \ref{eq:elD2C} nur von $el_D$ abhängt, kann diese unabhängig von \ref{eq:azD2C} umgestellt werden. Dazu werden die in Gleichung \ref{eq:elD2C} stehenden Sinus- und Kosinusfunktionen durch die gleichen Tangensfunktionen ausgedrückt. Dazu werden zunächst Sinus als auch Kosinus als Funktion der halben Winkel ausgedrückt
\begin{equation}
\sin\left( el_D \right) = \sin\left( \frac{el_D}{2} \right)\cos\left( \frac{el_D}{2} \right)+\cos\left( \frac{el_D}{2} \right)\sin\left( \frac{el_D}{2} \right)=2\sin\left( \frac{el_D}{2} \right)\cos\left( \frac{el_D}{2} \right)
\end{equation}
\begin{equation}
\cos\left( el_D \right) = \cos\left( \frac{el_D}{2} \right)\cos\left( \frac{el_D}{2} \right)+\sin\left( \frac{el_D}{2} \right)\sin\left( \frac{el_D}{2} \right)=\cos^2\left( \frac{el_D}{2} \right)-\sin^2\left( \frac{el_D}{2} \right).
\end{equation}
Teilt man die Gleichungen durch $1=\sin^2\left( \frac{el_D}{2} \right)+\cos^2\left( \frac{el_D}{2} \right)$ erhält man
\begin{equation}
\sin\left( el_D \right)=\frac{2\sin\left( \frac{el_D}{2} \right)\cos\left( \frac{el_D}{2} \right)}{\sin^2\left( \frac{el_D}{2} \right)+\cos^2\left( \frac{el_D}{2} \right)}
\end{equation}
und
\begin{equation}
\cos\left( el_D \right)=\frac{\cos^2\left( \frac{el_D}{2} \right)-\sin^2\left( \frac{el_D}{2} \right)}{\sin^2\left( \frac{el_D}{2} \right)+\cos^2\left( \frac{el_D}{2} \right)}.
\end{equation}
Ersetzt man $\sin\left(\frac{el_D}{2}\right)$ durch $\cos\left(\frac{el_D}{2}\right)\tan\left(\frac{el_D}{2}\right)$ ergeben sich folgende Gleichungen:
\begin{equation}
\sin\left(el_D\right)=\frac{2t}{1+t^2}
\label{eq:sint}
\end{equation}
\begin{equation}
\cos\left(el_D\right)=\frac{1-t^2}{1+t^2},
\label{eq:cost}
\end{equation}
wobei $t$ der Tangens des halben Winkels ist
\begin{equation}
t=\tan\left(\frac{el_D}{2}\right).
\label{eq:t}
\end{equation}
Somit ergibt sich für Gleichung \ref{eq:elD2C} folgender Ausdruck
\begin{equation}
\sin\left(el_C\right)=\frac{2t\cos\left(el_0\right)\cos\left(az_0\right)+\left(1-t^2\right)\sin\left(el_0 \right)}{1+t^2}.
\end{equation}
beziehungsweise
\begin{equation}
t^2\left(\sin(el_C)+\sin(el_0)\right)-t\left(2\cos(az_0)\cos(el_0)\right)+\sin(el_C)-\sin(el_0)=0
\end{equation}
Aufgelöst nach $t$ ergeben sich folgende Lösungen
\begin{equation}
t_{1,2}=\frac{\cos\left(el_0\right)\cos\left(az_0\right)\pm\sqrt{\cos^2\left(el_0\right)\cos^2\left(az_0\right)+\sin^2\left(el_0\right)-\sin^2\left(el_C\right)}}{\sin\left(el_C\right)+\sin\left(el_0\right)}
\label{eq:t12}
\end{equation}
Um das relevante der beiden Lösungen herauszubekommen setzt man $az_0=el_0=0$
\begin{equation}
t_{1,2}=\frac{1\pm\sqrt{1+0-\sin^2\left(el_C\right)}}{\sin\left(el_C\right)+0}=\frac{1\pm\cos\left(el_C\right)}{\sin\left(el_C\right)}
\end{equation}
was mit den Gleichungen \ref{eq:sint} und \ref{eq:cost} zu 
\begin{equation}
t_{1,2}(el_D)=\frac{1+t^2(el_C)\pm(1-t^2(el_C)}{2t(el_C)}=\left\{\begin{array}{lr}
\frac{1}{t(el_C)} & "+" \\
t(el_C) & - \\
\end{array}\right
\end{equation}
wird. Da für $az_0=el_0=0$ $el_D$ und $el_C$ identisch sind, kommt hier nur die Lösung mit dem negativen Vorzeichen in Betracht. Ersetzt man den linken Teil der Formel \ref{eq:t12} durch \ref{eq:t} und löst sich die Gleichung nach $el_D$ auf, so erhält man
\begin{equation}
el_D=2\arctan\left(\frac{\cos(el_0)\cos(az_0)-\sqrt{\cos^2(el_0)\cos^2(az_0)+\sin^2(el_0)-\sin^2(el_C)}}{\sin(el_C)+\sin(el_0)}\right)
\label{eq:elC2D}
\end{equation}
%Betrachtet man den Wertebereich der Funktion stellt man fest, dass
%Zur Untersuchung der Azimutabhängigkeit werden aus Gründen der Übersichtlichkeit folgende Terme substituiert
%\begin{equation}
%\cos\left(el_D\right)\cos\left(az_0\right)\cos\left(el_0\right)-\sin\left(el_D\right)\sin\left(el_0\right)=a
%\end{equation}
%\begin{equation}
%\sin\left(az_0\right)\cos\left(el_D\right)=b
%\end{equation}
Zur Untersuchung der Azimutabhängigkeit geht man von Formel \ref{eq:azD2C} aus und löst den arctan2 auf
\begin{equation}
\tan\left(az_C\right)=\frac{a\sin\left(az_D\right)+b\cos\left(az_D\right)}{a\cos\left(az_D\right)-b\sin\left(az_D\right)}
\end{equation}
ausdrücken. Teilt man Zähler und Nenner durch $\cos\left(az_D\right)$ so erhält man
\begin{equation}
\tan\left(az_C\right)=\frac{a\tan\left(az_D\right)+b}{a-b\tan\left(az_D\right)}.
\end{equation}
Aufgelöst nach $\tan\left(az_D\right)$ erhält man
\begin{equation}
\tan\left(az_D\right)=\frac{a\tan\left(az_C\right)-b}{a+b\tan\left(az_C\right)}
\end{equation}
beziehungsweise
\begin{equation}
\tan\left(az_D\right)=\frac{a\sin\left(az_C\right)-b\cos\left(az_C\right)}{a\cos\left(az_C\right)+b\sin\left(az_C\right)}
\end{equation}
Für $az_D$ erhält man somit 
\begin{equation}
az_D=\arctan 2\left(
%\sin(az_C)(\cos(el_C)\cos(az_0)\cos(el0)-\sin(el_C)\sin(el_0))-\cos(az_C)\sin(az_0)\cos(el0),
%\cos(az_C)(\cos(el_C)\cos(az_0)\cos(el0)-\sin(el_C)\sin(el_0))+\sin(az_C)\sin(az_0)\cos(el0))
a\sin\left(az_C\right)-b\cos\left(az_c\right),a\cos\left(az_C\right)+b\sin\left(az_c\right)\right)
\label{eq:azC2D}
\end{equation}
wobei hier noch $el_D$ in $a$ und $b$ durch den Ausdruck \ref{eq:elC2D} ersetzt werden muss.

\section{Erweiterung des Modells auf vier Parameter}
\label{se:4par}
Unter der Annahme, dass die Skalen des Drivesystems einen Offset haben, also das der Wertebereich von $el_D$ beispielshalber von $1^{\circ}$ bis $91^{\circ}$ läuft, kann das Modell mit zwei additiven Konstanten $el_1$ und $az_1$ erweitert werden. Das bedeutet, das bei der Vorhersage der CCD-Koordinaten \ref{eq:elD2C} und \ref{eq:azD2C} ein konstanter Offset zu den Argumenten hinzugefügt wird:
\begin{equation}
el_C^{4par}=el_C(el_D+el_1,az_D+az_1)
\label{eq:elD2C4}
\end{equation}
\begin{equation}
az_C^{4par}=az_C(el_D+el_1,az_D+az_1)
\label{eq:azD2C4}
\end{equation}
wohingegen bei der Vorhersage der Drive-Koordinaten nur ein konstanter Offset zu den Funktionen \ref{eq:elC2D} und \ref{eq:azC2D} addiert wird.
\begin{equation}
el_D^{4par}=el_D(el_C,az_C)+el_1
\label{eq:elC2D4}
\end{equation}
\begin{equation}
az_D^{4par}=az_D(el_C,az_C)+az_1
\label{eq:azC2D4}
\end{equation}

% usw.

%-Anhang----------------------------------------------------------

\appendix

%-Literaturverzeichnis--------------------------------------------

%\nocite{*}
%die Verwendung von bibtex ist Pflicht!!!
\bibliography{bibliography}
\bibliographystyle{plaindin}        %bzw. unsrtdin, alphadin, abbrvdin

%-Kapitel des Anhangs---------------------------------------------

%\chapter{\Alph{Anhang}Anhang}
\section*{Ergebnisse der Fits}
\subsection*{Vorhersage der Drivekoordinaten im Zwei-Parameter-Modell}
\begin{table}[htbp]
\centering
\begin{tabular}{rcl}
\toprule
$el_0$ &=& $(-1,20\pm0,57)^{\circ}$\\
$az_0$ &=& $(12,04\pm0,53)^{\circ}$\\
$\sigma$ &=& $0,13^{\circ}$\\
$\rho_{el_0,az_0}$ &=& $0,2618$\\
\bottomrule
\end{tabular}
\caption{Die für das Pointingmodell mit zwei Parametern bestimmten Werte, wobei die Drivekoordinaten in Abhängigkeit der CCD-Koordinaten vorhergesagt wurden.}
\label{tab:C2D-}
\end{table}
\subsection*{Vorhersage der CCD-Kordinaten im Zwei-Parameter-Modell}
\begin{table}[htbp]
\centering
\begin{tabular}{rcl}
\toprule
$el0$ &=& $(-1,20\pm0,55)^{\circ}$\\
$az0$ &=& $(12,05\pm0,52)^{\circ}$\\
$\sigma$ &=& $0,13^{\circ}$\\
$\rho_{el_0,az0}$ &=& $-0,00401$\\
\bottomrule
\end{tabular}
\caption{Die für das Pointingmodell mit zwei Parametern bestimmten Werte, wobei die CCD-Koordinaten in Abhängigkeit der Drive-Koordinaten vorhergesagt wurden.}
\label{tab:D2C-}
\end{table}
\subsection*{Vorhersage der Drivekoordinaten im Vier-Parameter-Modell}
\begin{table}[htbp]
\centering
\begin{tabular}{rcl}
\toprule
$el_0$ &=& $(-1,1\pm 4,9)^{\circ}$\\
$az_0$ &=& $(12,0\pm3,6)^{\circ}$\\
$el_1$ &=& $(0,03\pm 5,96)^{\circ}$\\
$az_1$ &=& $(-0,1\pm4,2)^{\circ}$\\
$\sigma$ &=& $0,13^{\circ}$\\
$\rho_{el_0,az_0}$ &=& $-0,737$\\
$\rho_{el_0,el_1}$ &=& $-0,9842$\\
$\rho_{az_0,az_1}$ &=& $-0,9038$\\
$\rho_{el_0,az_1}$ &=& $0,4214$\\
$\rho_{az_0,el_1}$ &=& $0,8267$\\
$\rho_{el_1,az_1}$ &=& $-0,5417$\\
\bottomrule
\end{tabular}
\caption{Die für das Pointingmodell mit vier Parametern bestimmten Werte, wobei die Drive-Koordinaten in Abhängigkeit der CCD-Koordinaten vorhergesagt wurden.}
\label{tab:C2D4-}
\end{table}
\newpage
\subsection*{Vorhersage der CCD-Koordinaten im Vier-Parameter-Modell}
\begin{table}[htbp]
\centering
\begin{tabular}{rcl}
\toprule
$el_0$ &=& $(-1\pm 78)^{\circ}$\\
$az_0$ &=& $(12,1\pm2,1)^{\circ}$\\
$el_1$ &=& $(0,04\pm 79,29)^{\circ}$\\
$az_1$ &=& $(-0,06\pm3,61)^{\circ}$\\
$\sigma$ &=& $0,13^{\circ}$\\
$\rho_{el_0,az_0}$&=& $-0,1681$\\
$\rho_{el_0,el_1}$&=& $-0,9999$\\
$\rho_{az_0,az_1}$&=& $-0,9526$\\
$\rho_{el_0,az_1}$&=& $-0,003998$\\
$\rho_{az_0,el_1}$&=& $0,1756$\\
$\rho_{el_1,az_1}$&=& $-0,003827$\\
\bottomrule
\end{tabular}
\caption{Die für das Pointingmodell mit vier Parametern bestimmten Werte, wobei die CCD-Koordinaten in Abhängigkeit der Drive-Koordinaten vorhergesagt wurden.}
\label{tab:D2C4-}
\end{table}
%\include{appendixB}
%usw.

%-Abkuerzungen*---------------------------------------------------

%\include{abbreviations}

%-Danksagung*-----------------------------------------------------

%\include{acknowledgement}

%-Lebenslauf*-----------------------------------------------------

%\include{cv}

%-Selbst�ndigkeiterkl�rung--------------------------------------

\chapter*{Selbst\"andigkeitserkl\"arung}

Text der Selbst\"andigkeitserkl\"arung.


%-----------------------------------------------------------------

\end{document}
