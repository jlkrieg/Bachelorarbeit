\chapter{Das Cherenkov Telescope Array}
Mit dem Bau des Cherenkov Telescope Arrays (CTA) werden verschiedene Ziele verfolgt:
1.Verbesserung des Senstitiviteatslevels um eine Grossenordnung auf 1TeV
2.Erhoehung der Detektionsflaeche/Photonenrate->Zugang zu kurzeitigen Ereignissen
3.Erhoehung der Winkelaufloesung/des Sichtfeldes
4.Energieabdeckung von 20GeV bis 300TeV
5.Verbesserung des Vermessungsfaehigkeit, Ueberwachungsfaehigkeit und Flexibilitaet->gleichzeitige Beobachtung von Objekten in verschiedenen Feldern
6.Datenproduktion und Tools auf fuer nicht Experten
7.Abdeckung des gesamten Himmels (nord+sued)


\section{Teleskoptypen}
Fuer das CTA werden drei verschieden grosse Teleskopgroessen entwickelt.
\subsection{Small-Sized-Telescope (SST)}
\subsection{Medium-Sized-Telescope (SST)}
\subsection{Large-Sized-Telescope (SST)}

\section{Array Konzept}
Um sowohl die suedliche als auch die noerdliche Hemnisphaere abzudecken, wird das CTA in der Atacamawueste in Chile und auf der zu Spanien gehoerenden Insel La Palma errichtet.
\subsection{Chile}
\subsection{Spanien}

\section{Prototyp in Adlershof}
In Adlershof wurde 2012 vom DESY ein Prototyp des MSTs errichtet um den mechanischen Aufbau zu testen, Pointingmodelle zu entwickeln und um die einzelnen Spiegel zu testen und auszurichten.
