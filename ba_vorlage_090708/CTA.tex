\chapter{Das Cherenkov Telescope Array}
Mit dem Bau des Cherenkov Telescope Arrays (CTA) werden verschiedene Ziele verfolgt:
\begin{itemize}

\item Verbesserung des Senstitiviteatslevels um eine Grossenordnung auf 1TeV
\item Erhoehung der Detektionsflaeche/Photonenrate->Zugang zu kurzeitigen Ereignissen
\item Erhoehung der Winkelaufloesung/des Sichtfeldes
\item Energieabdeckung von 20GeV bis 300TeV
\item Verbesserung des Vermessungsfaehigkeit, Ueberwachungsfaehigkeit und Flexibilitaet->gleichzeitige Beobachtung von Objekten in verschiedenen Feldern
\item Datenproduktion und Tools auf fuer nicht Experten
\item Abdeckung des gesamten Himmels (nord+sued)
\end{itemize}

\section{Design-Konzept}
Um sowohl die suedliche als auch die noerdliche Hemnisphaere abzudecken, wird das CTA in der Atacamawueste in Chile und auf der zu Spanien gehoerenden Insel La Palma errichtet.


\section{Prototyp in Adlershof}
In Adlershof wurde 2012 vom DESY ein Prototyp des MSTs errichtet um den mechanischen Aufbau zu testen, Pointingmodelle zu entwickeln und um die einzelnen Spiegel zu testen und auszurichten.

\subsection{Kameras des MST}
Der Prototyp des MST besitzt drei Kameras in der Mitte des Reflektors. Die Sky-CCD, für die im hier folgenden ein Pointingmodell entwickelt wird, ist schräg montiert, sodass sie am Detektorarm vorbei guckt um Bilder des Nachthimmels aufzunehmen. Aus diesen Bildern lassen sich mithilfe der Astrometry-Software die Koordinaten der Kamera bestimmen, die als die wahren Koordinaten angenommen werden.

\subsection{Koordinatens des MST}
Als geeignetes Koordinatensystem für den Betrieb eines Teleskops erweist sich ein mit zwei Winkeln zu beschreibendes System, das den Kugelkoordinaten ähnelt. Der Azemutwinkel behält seinen Namen und zeigt in der Regel bei $0^\circ$ in Richtung Norden. Der Polarwinkel behält ebenfalls seine Bedeutung und wird Elevation genannt.