\chapter{Fazit}
In dieser Arbeit wurden zunächst Pointingmodelle entwickelt, die den Winkel zwischen der optischen Achse und der Sky-CCD des MST-Prototypen korrigieren sollen. Im Gegensatz zu den meisten Pointingproblemen sind die Abweichungen hier so groß, dass es nicht mehr reicht, die Korrekturen als Differenzen zu entwickeln. so wurden Formeln entwickelt, die sowohl die Position der CCD in Abhängigkeit der Koordinaten des Telekops als auch die umgekehrte Abhängigkeit beschreiben. Auf der einen Seite wurde ein Modell entwickelt, dass mithilfe von zwei Parametern den Winkel zwischen der optischer Achse und der CCD-Kamera beschreibt und auf der anderen Seite ein Modell entwickelt, das zusätzlich noch eine eine mögliche Verschiebung der Skalen des Drive-Systems berücksichtigt. Um diese beiden Modelle auf Konsitenz zu überprüfen wurden diese auf einen Datensatz des Prototypen angewendet. Im Vergleich der beiden Modelle fiel auf, dass die beiden zusätzliche Parameter des Vier-Parameter-Modell stark mit den beiden anderen Parametern korreliert sind und das Modell somit keine Verbesserung gegenüber dem Zwei-Parameter-Modell bringt. Beim Zwei-Parameter-Modell fällt zunächst auf, dass es die Abweichungen zwischen Drive-System und CCD-Kamara größtenteils korrigiert. Allerdings fällt bei der Betrachtung der Differenzen zwischen Modellwerten und gemessenen Werten auf, dass diese nicht zufällig streuen sondern systematisch abweichen. Das deutet darauf hin, dass das Zwei-Parameter-Modell noch nicht vollständig ist. Da die Differenzen nun aber klein sind, lassen sich die Differenzen isoliert korrigieren, so wei es Ruslan Konno \cite{Ruslan} für das MST bereits getan hat. Dabei hat er auch die Verkippung der Teleskopachse berücksichtigt, die den Großteil der verbliebenden Abweichungen korrigieren könnte.