\chapter{Fazit}
%In dieser Arbeit wurden zunächst Pointingmodelle entwickelt, die den Winkel zwischen der optischen Achse und der Sky-CCD des MST-Prototypen korrigieren sollen. Im Gegensatz zu den meisten Pointingproblemen sind die Abweichungen hier so groß, dass es nicht mehr reicht, die Korrekturen als Differenzen zu entwickeln. So wurden Formeln entwickelt, die sowohl die Position der CCD in Abhängigkeit der Koordinaten des Telekops als auch die umgekehrte Abhängigkeit beschreiben. Auf der einen Seite wurde ein Modell entwickelt, dass mithilfe von zwei Parametern den Winkel zwischen der optischer Achse und der CCD-Kamera beschreibt und auf der anderen Seite ein Modell entwickelt, das zusätzlich noch eine eine mögliche Verschiebung der Skalen des Drive-Systems berücksichtigt. Um diese beiden Modelle auf Konsistenz zu überprüfen wurden diese auf einen Datensatz des Prototypen angewendet. Im Vergleich der beiden Modelle fiel auf, dass die beiden zusätzliche Parameter des Vier-Parameter-Modell stark mit den beiden anderen beiden Parametern korreliert sind und das Modell somit keine Verbesserung gegenüber dem Zwei-Parameter-Modell bringt. Beim Zwei-Parameter-Modell fällt zunächst auf, dass es die Abweichungen zwischen Drive-System und CCD-Kamara größtenteils korrigiert. Allerdings fällt bei der Betrachtung der Differenzen zwischen Modellwerten und gemessenen Werten auf, dass diese nicht zufällig streuen sondern systematisch abweichen. Das deutet darauf hin, dass das Zwei-Parameter-Modell noch nicht vollständig ist. Da die Differenzen nun aber klein sind, lassen sich die Differenzen isoliert korrigieren, so wie es Ruslan Konno \cite{Ruslan} für das MST bereits getan hat. Dabei hat er auch die Verkippung der Teleskopachse berücksichtigt, die den Großteil der verbliebenden Abweichungen korrigieren könnte.


Der Prototyp des MST ist im Zentrum des Reflektors mit drei CCD-Kameras ausgestattet. Da eine dieser Kameras schräg zur optischen Achse des Teleskops ausgerichtet ist, entstehen große Abweichungen zwischen den Koordinaten des Teleskops und dieser Kamera. Um diese Abweichungen zu korrigieren wurden zwei Pointingmodelle entwickelt. Das Zwei-Parameter-Modell berücksichtigt nur den Winkel zwischen Kamera und Teleskop und das Vier-Parameter-Modell berücksichtigt zusätzlich noch die Verschiebung der Skalen des Teleskops.\\
Diese beiden Modelle wurden auf Daten des Teleskops angewendet, um die Konsistenz dieser Modelle sowie die Verbesserung durch die zusätzlichen Parameter zu überprüfen. Vergleicht man die beiden Modelle, stellt man fest, dass sie ähnliche Ergebnisse liefern. Das liegt zum einen an der Korrelation der Parameter im erweiterten Modell und zum anderen daran, dass die beiden zusätzlichen Parameter innerhalb ihrer Unsicherheiten verschwinden. Die Verschiebung der Skalen spielt also nur eine untergeordnetet Rolle. Mit den optimalen Parametern für den Winkel in Azimutrichtung $$az_0=(12,0\pm0,5)^{\circ}$$ und in Elevationsrichtung $$el_0=(-1,2\pm0,6)^{\circ}$$ korrigiert das Zwei-Parameter-Modell die Abweichungen größtenteils.\\
Bei genauerer Betrachtung der Differenzen zwischen den gemessenen Werten und denen des Pointingmodells fällt auf, dass diese nicht zufällig streuen, sondern systematisch abweichen. Daraus lässt sich schließen, dass das Zwei-Parameter-Modell nicht vollständig ist. Da die verbleibenden Abweichungen nun aber deutlich kleiner sind, können diese isoliert betrachtet werden, wie es beispielsweise Ruslan Konno \cite{Ruslan} bereits getan hat.
